\documentclass{ximera}

\graphicspath{     %% setup a global graphics path
{./}               %% look in the same-level directory
{./graphics/}      %% look in graphics
{../graphics/}     %% look up one directory, then in graphics
%{../../graphics/} %% look up two directories, then in graphics
}
% Extra pakketen



\usepackage{fontawesome5}
\usepackage{xcolor}
\usepackage{physics}
%\usepackage{siunitx}
\usepackage{lipsum}
%\usepackage{circuitikz}
\usepackage{csvsimple}
\usetikzlibrary{3d} % for canvas is
\usetikzlibrary{decorations.pathmorphing,decorations.markings} % for random steps & snake
\usepackage{datatool}
\usepackage{pgffor}
\usepackage{eurosym}
\usepackage{algorithmic}
\usepackage{pythontex} 
\usepackage{xparse}
%Extra commando's

\newcommand{\datum}{\today}
\newcommand{\nivcol}{black}
\newcommand{\iniv}{0} % index van het beheersingsniveau














% In plaats van grijs achtergrond, een rode achtergrond

%\renewcommand{\important}[1]{\ensuremath{\fcolorbox{red}{red!50}{$#1$}}}


% Uitbreidingwet zijn extra's die wel redelijkerwijze tot de leerstof van bv meer geavanceerde versies kunnen behoren voor de wetenschappelijke richtingen

\ifdefined\isXourse
   \ifdefined\xmuitweidingwet
   \else
       \def\xmnouitweidingwet{false}
   \fi
\fi

\ifdefined\xmuitweidingwet
\newcounter{xmuitweidingwet}  % anders error undefined ...  
\includecomment{xmuitweidingwet}
\else
\newtheoremstyle{dotless}{}{}{}{}{}{}{ }{}
\theoremstyle{dotless}
\newtheorem*{xmuitweidingwetnofrills}{}   % nofrills = no accordion; gebruikt dus de dotless theoremstyle!

\newcounter{xmuitweidingwet}
\newenvironment{xmuitweidingwet}[1][ ]%
{% 
	\refstepcounter{xmuitweidingwet}
	\begin{accordion}\begin{accordion-item}[\color{magenta}{{ $\ggg$ UITBREIDING WETENSCHAPPEN}} \arabic{xmuitweidingwet}: #1]%
			\begin{xmuitweidingwetnofrills}%
			}
			{\end{xmuitweidingwetnofrills}\end{accordion-item}\end{accordion}}   
\fi


%uitbreiding wiskunde


\ifdefined\isXourse
   \ifdefined\xmuitweidingwis
   \else
       \def\xmnouitweidingwis{false}
   \fi
\fi

\ifdefined\xmuitweidingwis
\newcounter{xmuitweidingwis}  % anders error undefined ...  
\includecomment{xmuitweidingwis}
\else
\newtheoremstyle{dotless}{}{}{}{}{}{}{ }{}
\theoremstyle{dotless}
\newtheorem*{xmuitweidingwisnofrills}{}   % nofrills = no accordion; gebruikt dus de dotless theoremstyle!

\newcounter{xmuitweidingwis}
\newenvironment{xmuitweidingwis}[1][ ]%
{% 
	\refstepcounter{xmuitweidingwis}
	\begin{accordion}\begin{accordion-item}[\color{purple}{{$\infty$ UITBREIDING WISKUNDE}} \arabic{xmuitweidingwis}: #1]%
			\begin{xmuitweidingwisnofrills}%
			}
			{\end{xmuitweidingwisnofrills}\end{accordion-item}\end{accordion}}   
\fi


% Oefening enkel voor wetenschappen en wiskunde


\newenvironment{oefwet}[1]
{   \textcolor{teal}{
    $\ggg WETENSCHAPPEN$ \hrulefill \\
        	#1 
        	}
}

% % afbeelding vak
\newcommand{\wis}[1]{$\infty$ #1}
\newcommand{\wet}[1]{ #1}
\newcommand{\fys}[1]{$\nabla$ #1}
\newcommand{\info}[1]{$\vdots$ #1}

\newcommand{\arch}[1]{$\measuredangle
$ #1}

\newcommand{\eco}[1]{\officialeuro #1}

%tabel grootheid en eenheid
% \newenvironment{eenheid}[4]
% {
% \renewcommand{\arraystretch}{2}

%     \begin{table}[h]
%         \centering
%         \begin{tabular}{|c|c||c|c|}
%         \hline
%              \large{Grootheid} &  \large{Symbool} & \large{Eenheid} & 
%              \large{Symbool} \\
%              \hline \hline
%              \cellcolor{teal!50}\textbf{\LARGE{#1}} & \cellcolor{teal!50}\textbf{\LARGE{#2}} & \cellcolor{teal!50}\textbf{\LARGE{#3}} & \cellcolor{teal!50}\textbf{\LARGE{#4}} \\
%              \hline
%         \end{tabular}
%     \end{table}

% \renewcommand{\arraystretch}{1}
% }
%Leerplaandoelstelling bepalen
\newcommand{\doel}[1]
  {
    \DTLsetseparator{;}  %csv-scheidingsteken
    
    \def\mydb{doel}
    \edef\mydb{\mydb{}#1}
    \DTLloaddb{\mydb}{../Data/LPD_wiskunde.csv} % laad csv-bestand 
    \DTLassignfirstmatch{\mydb}{REF}{#1} %lees een rij met kolommen
    {\REF=REF,\LPD=LPD,\OND=ONDERWERP,\OMS=OMSCHRIJVING,\NIV=BEHEERSINGSNIVEAU}

    \def\nivM{M} %memoriseren index 1
    \def\nivB{B} %begrijpen index 2
    \def\nivT{T} %toepassen index 3
    \def\nivA{A} %analyseren index 4
    \def\nivE{E} %evalueren index 5
    \def\nivC{C} %creeren index 6
    \ifx\nivM\NIV \renewcommand{\nivcol}{gray} \renewcommand{\iniv}{1}
    \else
            \ifx\nivB\NIV \renewcommand{\nivcol}{blue} \renewcommand{\iniv}{2}
            \else 
                \ifx\nivT\NIV \renewcommand{\nivcol}{green} \renewcommand{\iniv}{3}
                \else
                    \ifx\nivA\NIV \renewcommand{\nivcol}{orange} \renewcommand{\iniv}{4}
                    \else
                        \ifx\nivE\NIV \renewcommand{\nivcol}{red} \renewcommand{\iniv}{5}
                        \else 
                            \ifx\nivC\NIV \renewcommand{\nivcol}{violet} \renewcommand{\iniv}{6}
                            \else fout \fi
                        \fi
                    \fi
                \fi
            \fi 
    \fi
    

    
    
    
    
   \begin{tikzpicture}[scale=0.25]
        % Static part
        \draw[draw=white,fill=violet,thick] (-45:2) -- (-45:3.5) arc(-45:0:3.5) -- (0:2) arc(0:-45:2) -- cycle ;
        \draw[draw=white,fill=red,thick] (45:2cm)-- (45:3.5cm) arc (45:0:3.5) -- (0:2cm) arc (0:45:2);
        \draw[draw=white,fill=orange,thick] (90:2cm)-- (90:3.5cm) arc (90:45:3.5) -- (45:2cm) arc (45:90:2);
        \draw[draw=white,fill=green,thick] (135:2cm)-- (135:3.5cm) arc (135:90:3.5) -- (90:2cm) arc (90:135:2);
        \draw[draw=white,fill=blue,thick] (180:2cm)-- (180:3.5cm) arc (180:135:3.5) -- (135:2cm) arc (135:180:2);
        \draw[draw=white,fill=gray,thick] (225:2cm)-- (225:3.5cm) arc (225:180:3.5) -- (180:2cm) arc (180:225:2);
        \draw[draw=white,fill=\nivcol,thick] ({225-(\iniv)*45}:2cm)-- ({225-(\iniv)*45}:5cm) arc ({225-(\iniv)*45}:{225-(\iniv-1)*45}:5) -- ({225-(\iniv-1)*45}:2cm) arc ({225-(\iniv-1)*45}:{225-(\iniv)*45}:2);
        
        % Speedometer needle
        %\pgfmathsetmacro{\col}{ifthenelse(\NIV==T,"orange",ifthenelse(\NIV==A,"blue"))}
        \node at (0,0) {\textcolor{\nivcol}{\huge{\textbf{\NIV}}}};
        \node at (6,5) [below right,text width=0.80\textwidth] {\OMS}
        \end{tikzpicture} \\
        
    %Aanduidend dat de doelsteling werd gebruikt
    \DTLgetrowindex{\rowidx}{\mydb}{\dtlcolumnindex{\mydb}{REF}}{#1}%
    \dtlgetrow{\mydb}{\rowidx}
    \dtlupdateentryincurrentrow{XIMERA_USE}{Ja}
    \dtlupdateentryincurrentrow{XIMERA_UPDATE}{\datum}
    \dtlrecombine
    \DTLsavedb{\mydb}{LPD_wiskunde2.csv}
  }

\newcommand{\cocalc}{\includegraphics[scale=0.05]{../pictures/cocalc_icon.png}}              %afbeelding cocalc (verwijst naar voorbeelden/oefening in, cocalc

\newcommand{\pcode}[1]{\texttt{\textcolor{red}{#1}}} %python-commando's weergeven in deze stijl

\newcommand{\csv}[2]{\href{#1}{(\includegraphics[scale=0.05]{csv-icon.png} #2)}}

\newcommand{\fystip}{\begin{warning}
	Denk aan!
	\begin{itemize}
		\item correct notatie grootheden.
		\item bij een grootheid hoort een eenheid.
		\item resultaten in wetenschappelijke notatie.
		\item beduidende cijfers.
	\end{itemize}
\end{warning}}



% Instelling om stukjes code weer te geven
    \usepackage{listings}
    
    
    
    %New colors defined below
    \definecolor{codegreen}{rgb}{0,0.6,0}
    \definecolor{codegray}{rgb}{0.5,0.5,0.5}
    \definecolor{codepurple}{rgb}{0.58,0,0.82}
    \definecolor{backcolour}{rgb}{0.95,0.95,0.92}
    
    %Code listing style named "mystyle"
    \lstdefinestyle{mystyle}{
      backgroundcolor=\color{backcolour}, commentstyle=\color{codegreen},
      keywordstyle=\color{red},
      numberstyle=\tiny\color{codegray},
      stringstyle=\color{codepurple},
      basicstyle=\ttfamily\footnotesize,
      breakatwhitespace=false,         
      breaklines=true,                 
      captionpos=b,                    
      keepspaces=true,                 
      numbers=left,                    
      numbersep=5pt,                  
      showspaces=false,                
      showstringspaces=false,
      showtabs=false,                  
      tabsize=2
    }
    
    \lstset{style=mystyle}

%niveau oefeningen

%Specifiek element uit de lijst van kleuren




\newcommand{\oefniv}[1]
{
\ifnum #1=1 
    \def\mycolor{gray}
\else 
    \ifnum #1=2
        \def\mycolor{blue}
    \else
        \ifnum #1=3
            \def\mycolor{green}
        \else
            \ifnum #1=4 
                \def\mycolor{orange}
            \else
                \ifnum #1=5 
                    \def\mycolor{red}
                \else
                    \ifnum #1=6 
                        \def\mycolor{violet}
                    \else
                        \def\mycolor{black}
                    \fi
                \fi 
            \fi 
        \fi
    \fi 
\fi
\begin{tikzpicture}[scale=0.15]
    \draw[draw=white,fill=gray!50,thick] (0, 0) rectangle (2, 2);
    \draw[draw=white,fill=blue!50,thick] (2, 0) rectangle (4, 2);
    \draw[draw=white,fill=green!50,thick] (4, 0) rectangle (6, 2);
    \draw[draw=white,fill=orange!50,thick] (6, 0) rectangle (8, 2);
    \draw[draw=white,fill=red!50,thick] (8, 0) rectangle (10, 2);
    \draw[draw=white,fill=violet!50,thick] (10, 0) rectangle (12, 2);

    \draw[draw=white,fill=\mycolor,thick] ({(#1-1)*2}, 0) -- ({(#1-1)*2},4) -- ({(#1)*2},4) -- ({(#1)*2},0) -- cycle;
\end{tikzpicture}
\\
}

%eigenschappen veer

\usetikzlibrary{decorations.pathmorphing,calc,patterns}

\makeatletter

\def\pgfdecorationspringstraightlinelength{.5cm}
\def\pgfdecorationspringnumberofelement{8}
\def\pgfdecorationspringnaturallength{5cm}
\pgfkeys{%
  /pgf/decoration/.cd,
  spring straight line length/.code={%
    \pgfmathsetlengthmacro\pgfdecorationspringstraightlinelength{#1}},
  spring natural length/.code={%
    \pgfmathsetlengthmacro\pgfdecorationspringnaturallength{#1}},
  spring number of element/.store in=\pgfdecorationspringnumberofelement
}

\pgfdeclaredecoration{coil spring}{straight line}{%
  \state{straight line}[%
    persistent precomputation = {%
      % Compute the effective length of the spring (without the length
      % of the two straight lines): \pgfdecorationspringeffectivelength
      \pgfmathsetlengthmacro{\pgfdecorationspringeffectivelength}%
        {\pgfdecoratedpathlength-2*\pgfdecorationspringstraightlinelength}
      % Compute the effective length of one coil pattern:
      % \pgfdecorationspringeffectivelengthofonecoil
      \pgfmathsetlengthmacro{\pgfdecorationspringeffectivelengthofonecoil}%
        {\pgfdecorationspringeffectivelength/\pgfdecorationspringnumberofelement}
    },
    width = \pgfdecorationspringstraightlinelength,
    next state = draw spring]{%
      \pgfpathlineto{%
        \pgfqpoint{%
          \pgfdecorationspringstraightlinelength}{0pt}}
  }
  \state{draw spring}%
    [width=\pgfdecorationspringeffectivelengthofonecoil,
     repeat state=\pgfdecorationspringnumberofelement-1,next state=final]{%
       \pgfpathcurveto
         {\pgfpoint@onspringcoil{0    }{ 0.555}{1}}
         {\pgfpoint@onspringcoil{0.445}{ 1    }{2}}
         {\pgfpoint@onspringcoil{1    }{ 1    }{3}}
       \pgfpathcurveto
         {\pgfpoint@onspringcoil{1.555}{ 1    }{4}}
         {\pgfpoint@onspringcoil{2    }{ 0.555}{5}}
         {\pgfpoint@onspringcoil{2    }{ 0    }{6}}
       \pgfpathcurveto
         {\pgfpoint@onspringcoil{2    }{-0.555}{7}}
         {\pgfpoint@onspringcoil{1.555}{-1    }{8}}
         {\pgfpoint@onspringcoil{1    }{-1    }{9}}
       \pgfpathcurveto
         {\pgfpoint@onspringcoil{0.445}{-1    }{10}}
         {\pgfpoint@onspringcoil{0    }{-0.555}{11}}
         {\pgfpoint@onspringcoil{0    }{ 0    }{12}}
  }
  \state{final}{%
    \pgfpathlineto{\pgfpointdecoratedpathlast}
  }
}

\def\pgfpoint@onspringcoil#1#2#3{%
  \pgf@x=#1\pgfdecorationsegmentamplitude%
  \pgf@x=.5\pgf@x%
  \pgf@y=#2\pgfdecorationsegmentamplitude%
  \pgfmathparse{0.083333333333*\pgfdecorationspringeffectivelengthofonecoil}%
  \pgf@xa=\pgfmathresult pt
  \advance\pgf@x by#3\pgf@xa%
}


\makeatother

\tikzset{%
  Spring/.style = {%
    decoration = {%
      coil spring,
      spring straight line length = .5cm,
      % To be added
      spring natural length = #1,
      spring number of element = 8,
      amplitude=3mm},
    decorate,
    very thick},
  Spring/.default = {4cm}}


\addPrintStyle{..}





\begin{document}
    \author{Ingmar Herreman}
    \date{Januari 2024}
    \xmtitle{\fys{Oefeningen op de trillingsvergelijking}}{}
    
\begin{exercise} \oefniv{2}
	Vul steeds aan.

	\begin{question}

\wordChoice{
	\choice{de frequentie}
	\choice{de periode}
	\choice{de uitwijking}
	\choice[correct]{de amplitude}
	\choice{de pulsatie}
	\choice{de beginfase}
	\choice{de fase}
}
van een trillend voorwerp is de maximale uitwijking.

	\end{question}

	\begin{question}
		De tijd die een massa-veersysteem nodig heeft om één volledige trilling uit te voeren noemen we 
\wordChoice{
		\choice{de frequentie}
		\choice[correct]{de periode}
		\choice{de uitwijking}
		\choice{de amplitude}
		\choice{de pulsatie}
		\choice{de beginfase}
		\choice{de fase}
}

	\end{question}


	\begin{question}
		Wanneer een trillend voorwerp zich op $t=0 \ s$ in de evenwichtstand bevindt, dan is de \wordChoice{
	\choice{de frequentie}
	\choice{de periode}
	\choice{de uitwijking}
	\choice{de amplitude}
	\choice{de pulsatie}
	\choice[correct]{de beginfase}
	\choice[correct]{de fase}
}

	\end{question}


	\begin{question}
		Hoe kleiner de periode van een harmonische trilling, hoe sneller de trillingen elkaar opvolgen hoe groter \wordChoice{
	\choice[correct]{de frequentie}
	\choice{de periode}
	\choice{de uitwijking}
	\choice{de amplitude}
	\choice[correct]{de pulsatie}
	\choice{de beginfase}
	\choice{de fase}
}.

	\end{question}

	\begin{question} 
		Het argument van de sinus noemen we 
\wordChoice{
	\choice{de frequentie}
	\choice{de periode}
	\choice{de uitwijking}
	\choice{de amplitude}
	\choice{de pulsatie}
	\choice{de beginfase}
	\choice[correct]{de fase}
}

	\end{question}

		\begin{question}
			Wanneer de beginfase gelijk is aan $\frac{\pi}{2}$, dan is \wordChoice{
	\choice{de frequentie}
	\choice{de periode}
	\choice[correct]{de uitwijking}
	\choice{de amplitude}
	\choice{de pulsatie}
	\choice{de beginfase}
	\choice{de fase}
} op $t=0 \ s$ maximaal.

		\end{question}
\end{exercise}

\begin{exercise} \oefniv{3}
    Bepaal de periode en de amplitude van de onderstaande trilling.
    
    \begin{image}[0.5\textwidth]
        \begin{tikzpicture}
            \begin{axis}[restrict y to domain=-20:1000,
                    ylabel={$x(m)$},
                    xlabel={$t(s)$},
                    %axis equal image=true,
                    grid,
                    grid style={gray!50},
                    grid=both,
                    axis y line=center,
                    axis x line=middle, 
                    axis on top=true,
                    xmin=-0.001,
                    xmax=0.45,
                    ymin=-0.1,
                    ymax=0.60,
                    ytick={0.25,0.50},
                    xtick={\empty},
                    x=15 cm,y=7.5 cm]
                \addplot[red,domain=0:1,smooth,samples=500,line width=0.5 mm] {0.25*sin(deg(2*pi*x/0.2))+0.25};
                \draw[red,fill=red] (axis cs: 0.15,0) circle (1 mm) node[below] {$0.15$};
            \end{axis}
        \end{tikzpicture}
    \end{image}

    \begin{oplossing}
        $T=0,020 \ s$ \\
        $A=\frac{0,50}{2}=0,25 \ m$
    \end{oplossing}

\end{exercise}

\begin{exercise}
    Hieronder is de grafiek gegeven van een harmonisch trillend voorwerp.
    
    \begin{image}[0.75\textwidth]
     
        \begin{tikzpicture}
            \begin{axis}[restrict y to domain=-200:1000,
                    ylabel={$x(cm)$},
                    xlabel={$t(s)$},
                    %axis equal image=true,
                    grid,
                    grid style={gray!50},
                    grid=both,
                    axis y line=center,
                    axis x line=middle, 
                    axis on top=true,
                    xmin=-0.1,
                    xmax=20,
                    ymin=-32,
                    ymax=32,
                    ytick={-30,-20,...,30},
                    xtick={0,2,...,18},
                    x=0.5 cm,y=0.1 cm]
                \addplot[red,domain=0:20,smooth,samples=500,line width=0.5 mm] {-30*sin(deg(2*pi*x/12)-360/12*4)};
            \end{axis}
        \end{tikzpicture}
    \end{image}

    \begin{question}
        Bepaal de amplitude , de periode en de frequentie van deze harmonische trilling.
        \begin{oplossing} \
            \begin{itemize}
                \item $A=0,30 \ m$
                \item $T=16-4=12 \ s$
                \item $f=\frac{1}{T}=\frac{1}{12} \ Hz$
            \end{itemize}
        \end{oplossing}
    \end{question}

\end{exercise}

\begin{exercise} \oefniv{3}
    Stel telkens de trillingsvergelijking op voor de onderstaande trillingen.
    \begin{question} 
        \begin{image}[0.5\textwidth]
            \begin{tikzpicture}
                \begin{axis}[restrict y to domain=-200:1000,
                            ylabel={$x(mm)$},
                            xlabel={$t(ms)$},
                            %axis equal image=true,
                            grid,
                            grid style={gray!50},
                            grid=both,
                            axis y line=center,
                            axis x line=middle, 
                            axis on top=true,
                            xmin=-0.1,
                            xmax=45,
                            ymin=-2.1,
                            ymax=2.2,
                            ytick={-1,-2,...,2},
                            xtick={0,10,...,40},
                            x=0.2 cm,y=1 cm]
                    \addplot[red,domain=0:50,smooth,samples=500,line width=0.5 mm] {2*sin(deg(2*pi*x/20))};
                \end{axis}
            \end{tikzpicture}
        \end{image}
    
        \begin{oplossing} \ 
            \begin{itemize}
                \item $A=2\cdot 10^{-3} \ m$
                \item $T=20\cdot 10^{-3} \ s \Leftrightarrow \omega=\frac{2\pi}{T}=100\pi s^{-1}$
                \item $y(t)=2\cdot 10^{-3} m \cdot \sin\left(100 \pi s^{-1}\cdot t\right)$
            \end{itemize}
        \end{oplossing}
    \end{question}
    
     \begin{question} 
        \begin{image}[0.5\textwidth]
            \begin{tikzpicture}
                \begin{axis}[restrict y to domain=-200:1000,
                        ylabel={$x(mm)$},
                        xlabel={$t(ms)$},
                        %axis equal image=true,
                        grid,
                        grid style={gray!50},
                        grid=both,
                        axis y line=center,
                        axis x line=middle, 
                        axis on top=true,
                        xmin=-0.1,
                        xmax=45,
                        ymin=-2.1,
                        ymax=2.2,
                        ytick={-1,-2,...,2},
                        xtick={0,10,...,40},
                        x=0.2 cm,y=1 cm]
                    \addplot[red,domain=0:50,smooth,samples=500,line width=0.5 mm] {2*sin(deg(2*pi*x/10))};
                \end{axis}
            \end{tikzpicture}
        \end{image}
        
        \begin{oplossing} \ 
            \begin{itemize}
                \item $A=2\cdot 10^{-3} \ m$
                \item $T=10\cdot 10^{-3} \ s \Leftrightarrow \omega=\frac{2\pi}{T}=200\pi s^{-1}$
                \item $y(t)=2\cdot 10^{-3} m \cdot \sin\left(200 \pi s^{-1}\cdot t\right)$
            \end{itemize}
        \end{oplossing}
    \end{question}
    
    \begin{question} \
        \begin{image}[0.5\textwidth]
            \begin{tikzpicture}
                \begin{axis}[restrict y to domain=-200:1000,
                        ylabel={$x(mm)$},
                        xlabel={$t(ms)$},
                        %axis equal image=true,
                        grid,
                        grid style={gray!50},
                        grid=both,
                        axis y line=center,
                        axis x line=middle, 
                        axis on top=true,
                        xmin=-0.1,
                        xmax=45,
                        ymin=-2.1,
                        ymax=2.2,
                        ytick={-1,-2,...,2},
                        xtick={0,10,...,40},
                        x=0.2 cm,y=1 cm]
                    \addplot[red,domain=0:50,smooth,samples=500,line width=0.5 mm] {1*sin(deg(2*pi*x/20))};
                \end{axis}
            \end{tikzpicture}
        \end{image}

        \begin{oplossing} \ 
            \begin{itemize}
                \item $A=1\cdot 10^{-3} \ m$
                \item $T=20\cdot 10^{-3} \ s \Leftrightarrow \omega=\frac{2\pi}{T}=100\pi s^{-1}$
                \item $y(t)=1\cdot 10^{-3} m \cdot \sin\left(100 \pi s^{-1}\cdot t\right)$
            \end{itemize}
        \end{oplossing}
    \end{question}
    
    \begin{question}
        \begin{image}[0.5\textwidth]
            \begin{tikzpicture}
                \begin{axis}[restrict y to domain=-200:1000,
                        ylabel={$x(mm)$},
                        xlabel={$t(ms)$},
                        %axis equal image=true,
                        grid,
                        grid style={gray!50},
                        grid=both,
                        axis y line=center,
                        axis x line=middle, 
                        axis on top=true,
                        xmin=-0.1,
                        xmax=45,
                        ymin=-2.1,
                        ymax=2.2,
                        ytick={-1,-2,...,2},
                        xtick={0,10,...,40},
                        x=0.2 cm,y=1 cm]
                    \addplot[red,domain=0:50,smooth,samples=500,line width=0.5 mm] {2*sin(deg(2*pi*x/20)+90)};
                \end{axis}
            \end{tikzpicture}
        \end{image}
        
        \begin{oplossing} \ 
            \begin{itemize}
                \item $A=2\cdot 10^{-3} \ m$
                \item $T=20\cdot 10^{-3} \ s \Leftrightarrow \omega=\frac{2\pi}{T}=100\pi s^{-1}$
                \item $\phi_0=-\frac{\pi}{2}$ (beweegt naar beneden)
                \item $y(t)=2\cdot 10^{-3} m \cdot \sin\left(100 \pi s^{-1}\cdot t-\frac{\pi}{2}\right)$
            \end{itemize}
        \end{oplossing}
    \end{question}
\end{exercise}



\begin{exercise} \oefniv{3}
		Vervolledig de onderstaande tabel. \ \\
		\begin{tabular}{|c|c|c|}
		    \hline
            $f$ & $\omega$ & $T$ \\ \hline
            $10 \ Hz$ & $\answer[onlinenoinput]{20\pi \ \frac{1}{s}}$ & $\answer[onlinenoinput]{0,1 \ s}$ \\ \hline
            $\answer[onlinenoinput]{50 \ Hz}$ & $\answer[onlinenoinput]{100\pi \ \frac{1}{s}}$ & $0,020 \ s$ \\ \hline
            $\answer[onlinenoinput]{15,91 \ Hz}$ & $100 \ \frac{1}{s}$ & $\answer[onlinenoinput]{0,063 \ s}$ \\ \hline
		\end{tabular}
\end{exercise}

\begin{exercise} \oefniv{2}
Een stemvork heeft een trillingsfrequentie van 440 Hz. Wat wil dat zeggen?
    \begin{oplossing}
        De benen van de stemvork bewegen 440 keren per seconde
    \end{oplossing}
\end{exercise}

\begin{exercise} \oefniv{3}
Stel de trillingsvergelijking op van deze trilling.
    \begin{image}[0.5\textwidth]
        \begin{tikzpicture}
            \begin{axis}[restrict y to domain=-200:1000,
                    ylabel={$x(m)$},
                    xlabel={$t(s)$},
                    %axis equal image=true,
                    grid,
                    grid style={gray!50},
                    grid=both,
                    axis y line=center,
                    axis x line=middle, 
                    axis on top=true,
                    xmin=-0.1,
                    xmax=16,
                    ymin=-10.5,
                    ymax=10.5,
                    ytick={-10,-5,...,10},
                    xtick={0,2,...,20},
                    x=0.5 cm,y=0.2 cm]
                \addplot[red,domain=0:50,smooth,samples=500,line width=0.5 mm] {10*sin(deg(2*pi*x/8))};
            \end{axis}
        \end{tikzpicture}
    \end{image}

    \begin{oplossing}
        $y(t)=10 \ m \cdot \sin \left(\frac{\pi}{4}s^{-1}\cdot t\right)$
    \end{oplossing}
\end{exercise}

\begin{exercise} \oefniv{4}
Een massa van $0,10 \ kg$ voert een harmonische trilling uit waarbij de amplitude $A = 0,05 \  m$ en de pulsatie  $\omega = 3,14 \ s^{-1}$ . 
    \begin{question}
        Bereken de periode en de frequentie van de trillende massa.
            \begin{oplossing}
                $T=2 \ s$ en $f=0,5 \ Hz$
            \end{oplossing}
    \end{question}

    \begin{question}
        Bereken op welk ogenblik de uitwijking respectievelijk  maximaal en nul is ?
            \begin{oplossing}
                $y(t)=0,05 \sin (3.14\cdot t)$
                \begin{itemize}
                    \item beginfase $\phi_0$ dus max. na $\frac{T}{4}=0.5 \ s$
                    \item beginfase $\phi_0$ dus nul na  $T=0 \ s$ en $\frac{T}{2}=1 \ s$
                \end{itemize}
            \end{oplossing}
    \end{question}
\end{exercise}

\begin{exercise} \oefniv{3}
Stel de trillingsvergelijking op voor een voorwerp dat een harmonische trilling uitvoert met amplitude $20,0 \ cm$ , frequentie $100 \ Hz$ en beginfase $\frac{\pi}{6}$
    \begin{oplossing}
        $\omega=2\pi \cdot f=200\pi \ s^{-1}$ \\
        $y(t)=20,0\cdot 10^{-2} \ m \ \sin\left(200\pi \ s^{-1}+\frac{\pi}{6}\right)$
    \end{oplossing}
\end{exercise}

\begin{exercise} \oefniv{4}
	We beschouwen twee trillende veren. Hun beweging wordt hieronder voorgesteld door middel van een $x\left(t\right)$-grafiek.
	
	\begin{image}[0.5\textwidth]
        \begin{tikzpicture} 
            \begin{axis}[restrict y to domain=-200:1000,
                    ylabel={$x(mm)$},
                    xlabel={$t(s)$},
                    title=Grafiek A,
                    %axis equal image=true,
                    grid,
                    grid style={gray!50},
                    grid=both,
                    axis y line=center,
                    axis x line=middle, 
                    axis on top=true,
                    xmin=-0.1,
                    xmax=40,
                    ymin=-155,
                    ymax=155,
                    ytick={-150,-100,...,150},
                    xtick={0,5,...,35},
                    x=0.2 cm,y=0.02 cm]
                \addplot[red,domain=0:50,smooth,samples=500,line width=0.5 mm] {100*sin(deg(2*pi*x/20)+90)};
                \addplot[blue,domain=0:50,smooth,samples=500,line width=0.5 mm] {50*sin(deg(2*pi*x/10)+90)};
            \end{axis}
        \end{tikzpicture}
    \end{image}

    \begin{image}[0.5\textwidth]
        \begin{tikzpicture}
            \begin{axis}[restrict y to domain=-200:1000,
                    ylabel={$x(mm)$},
                    xlabel={$t(s)$},
                    title=Grafiek B,
                    %axis equal image=true,
                    grid,
                    grid style={gray!50},
                    grid=both,
                    axis y line=center,
                    axis x line=middle, 
                    axis on top=true,
                    xmin=-0.1,
                    xmax=40,
                    ymin=-155,
                    ymax=155,
                    ytick={-150,-100,...,150},
                    xtick={0,5,...,35},
                    x=0.2 cm,y=0.02 cm]
                \addplot[red,domain=0:50,smooth,samples=500,line width=0.5 mm] {100*sin(deg(2*pi*x/10)-90)};
                \addplot[blue,domain=0:50,smooth,samples=500,line width=0.5 mm] {75*sin(deg(2*pi*x/10)-180)};
            \end{axis}
        \end{tikzpicture}
    \end{image}
	
    \begin{image}[0.5\textwidth]
        \begin{tikzpicture}
            \begin{axis}[restrict y to domain=-200:1000,
                    ylabel={$x(mm)$},
                    xlabel={$t(s)$},
                    title=Grafiek C,
                    %axis equal image=true,
                    grid,
                    grid style={gray!50},
                    grid=both,
                    axis y line=center,
                    axis x line=middle, 
                    axis on top=true,
                    xmin=-0.1,
                    xmax=40,
                    ymin=-155,
                    ymax=155,
                    ytick={-150,-100,...,150},
                    xtick={0,5,...,35},
                    x=0.2 cm,y=0.02 cm]
                \addplot[red,domain=0:50,smooth,samples=500,line width=0.5 mm] {100*sin(deg(2*pi*x/20)+90)};
                \addplot[blue,domain=0:50,smooth,samples=500,line width=0.5 mm] {50*sin(deg(2*pi*x/20)+90)};
            \end{axis}
        \end{tikzpicture}
    \end{image}

    \begin{image}[0.5\textwidth]
        \begin{tikzpicture}
            \begin{axis}[restrict y to domain=-200:1000,
                    ylabel={$x(mm)$},
                    xlabel={$t(s)$},
                    title=Grafiek D,
                    %axis equal image=true,
                    grid,
                    grid style={gray!50},
                    grid=both,
                    axis y line=center,
                    axis x line=middle, 
                    axis on top=true,
                    xmin=-0.1,
                    xmax=40,
                    ymin=-155,
                    ymax=155,
                    ytick={-150,-100,...,150},
                    xtick={0,5,...,35},
                    x=0.2 cm,y=0.02 cm]
                \addplot[red,domain=0:50,smooth,samples=500,line width=0.5 mm] {100*sin(deg(2*pi*x/20))};
                \addplot[blue,domain=0:50,smooth,samples=500,line width=0.5 mm] {100*sin(deg(2*pi*x/20)-180)};
            \end{axis}
        \end{tikzpicture}
    \end{image}

Welke van bovenstaande grafieken stemt overeen met de volgende beweringen?
    \begin{question}
        Twee veren trillen met dezelfde frequentie.
            \begin{oplossing}
                B,C,D
            \end{oplossing}
    \end{question}

    \begin{question}
        Twee veren trillen met eenzelfde frequentie, maar in verschillende zin.
            \begin{oplossing}
                D
            \end{oplossing}
    \end{question}

    \begin{question}
    Twee veren trillen met dezelfde frequentie maar verschillende amplitude.
        \begin{oplossing}
            B,C
        \end{oplossing}
    \end{question}

    \begin{question}
        Twee veren trillen met verschillende amplitude.  
            \begin{oplossing}
                A,B,C
            \end{oplossing}
    
\end{question}

\end{exercise}

\end{document}