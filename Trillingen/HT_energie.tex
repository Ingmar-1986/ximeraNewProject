\documentclass{ximera}

\graphicspath{     %% setup a global graphics path
{./}               %% look in the same-level directory
{./graphics/}      %% look in graphics
{../graphics/}     %% look up one directory, then in graphics
%{../../graphics/} %% look up two directories, then in graphics
}
% Extra pakketen



\usepackage{fontawesome5}
\usepackage{xcolor}
\usepackage{physics}
%\usepackage{siunitx}
\usepackage{lipsum}
%\usepackage{circuitikz}
\usepackage{csvsimple}
\usetikzlibrary{3d} % for canvas is
\usetikzlibrary{decorations.pathmorphing,decorations.markings} % for random steps & snake
\usepackage{datatool}
\usepackage{pgffor}
\usepackage{eurosym}
\usepackage{algorithmic}
\usepackage{pythontex} 
\usepackage{xparse}
%Extra commando's

\newcommand{\datum}{\today}
\newcommand{\nivcol}{black}
\newcommand{\iniv}{0} % index van het beheersingsniveau














% In plaats van grijs achtergrond, een rode achtergrond

%\renewcommand{\important}[1]{\ensuremath{\fcolorbox{red}{red!50}{$#1$}}}


% Uitbreidingwet zijn extra's die wel redelijkerwijze tot de leerstof van bv meer geavanceerde versies kunnen behoren voor de wetenschappelijke richtingen

\ifdefined\isXourse
   \ifdefined\xmuitweidingwet
   \else
       \def\xmnouitweidingwet{false}
   \fi
\fi

\ifdefined\xmuitweidingwet
\newcounter{xmuitweidingwet}  % anders error undefined ...  
\includecomment{xmuitweidingwet}
\else
\newtheoremstyle{dotless}{}{}{}{}{}{}{ }{}
\theoremstyle{dotless}
\newtheorem*{xmuitweidingwetnofrills}{}   % nofrills = no accordion; gebruikt dus de dotless theoremstyle!

\newcounter{xmuitweidingwet}
\newenvironment{xmuitweidingwet}[1][ ]%
{% 
	\refstepcounter{xmuitweidingwet}
	\begin{accordion}\begin{accordion-item}[\color{magenta}{{ $\ggg$ UITBREIDING WETENSCHAPPEN}} \arabic{xmuitweidingwet}: #1]%
			\begin{xmuitweidingwetnofrills}%
			}
			{\end{xmuitweidingwetnofrills}\end{accordion-item}\end{accordion}}   
\fi


%uitbreiding wiskunde


\ifdefined\isXourse
   \ifdefined\xmuitweidingwis
   \else
       \def\xmnouitweidingwis{false}
   \fi
\fi

\ifdefined\xmuitweidingwis
\newcounter{xmuitweidingwis}  % anders error undefined ...  
\includecomment{xmuitweidingwis}
\else
\newtheoremstyle{dotless}{}{}{}{}{}{}{ }{}
\theoremstyle{dotless}
\newtheorem*{xmuitweidingwisnofrills}{}   % nofrills = no accordion; gebruikt dus de dotless theoremstyle!

\newcounter{xmuitweidingwis}
\newenvironment{xmuitweidingwis}[1][ ]%
{% 
	\refstepcounter{xmuitweidingwis}
	\begin{accordion}\begin{accordion-item}[\color{purple}{{$\infty$ UITBREIDING WISKUNDE}} \arabic{xmuitweidingwis}: #1]%
			\begin{xmuitweidingwisnofrills}%
			}
			{\end{xmuitweidingwisnofrills}\end{accordion-item}\end{accordion}}   
\fi


% Oefening enkel voor wetenschappen en wiskunde


\newenvironment{oefwet}[1]
{   \textcolor{teal}{
    $\ggg WETENSCHAPPEN$ \hrulefill \\
        	#1 
        	}
}

% % afbeelding vak
\newcommand{\wis}[1]{$\infty$ #1}
\newcommand{\wet}[1]{ #1}
\newcommand{\fys}[1]{$\nabla$ #1}
\newcommand{\info}[1]{$\vdots$ #1}

\newcommand{\arch}[1]{$\measuredangle
$ #1}

\newcommand{\eco}[1]{\officialeuro #1}

%tabel grootheid en eenheid
% \newenvironment{eenheid}[4]
% {
% \renewcommand{\arraystretch}{2}

%     \begin{table}[h]
%         \centering
%         \begin{tabular}{|c|c||c|c|}
%         \hline
%              \large{Grootheid} &  \large{Symbool} & \large{Eenheid} & 
%              \large{Symbool} \\
%              \hline \hline
%              \cellcolor{teal!50}\textbf{\LARGE{#1}} & \cellcolor{teal!50}\textbf{\LARGE{#2}} & \cellcolor{teal!50}\textbf{\LARGE{#3}} & \cellcolor{teal!50}\textbf{\LARGE{#4}} \\
%              \hline
%         \end{tabular}
%     \end{table}

% \renewcommand{\arraystretch}{1}
% }
%Leerplaandoelstelling bepalen
\newcommand{\doel}[1]
  {
    \DTLsetseparator{;}  %csv-scheidingsteken
    
    \def\mydb{doel}
    \edef\mydb{\mydb{}#1}
    \DTLloaddb{\mydb}{../Data/LPD_wiskunde.csv} % laad csv-bestand 
    \DTLassignfirstmatch{\mydb}{REF}{#1} %lees een rij met kolommen
    {\REF=REF,\LPD=LPD,\OND=ONDERWERP,\OMS=OMSCHRIJVING,\NIV=BEHEERSINGSNIVEAU}

    \def\nivM{M} %memoriseren index 1
    \def\nivB{B} %begrijpen index 2
    \def\nivT{T} %toepassen index 3
    \def\nivA{A} %analyseren index 4
    \def\nivE{E} %evalueren index 5
    \def\nivC{C} %creeren index 6
    \ifx\nivM\NIV \renewcommand{\nivcol}{gray} \renewcommand{\iniv}{1}
    \else
            \ifx\nivB\NIV \renewcommand{\nivcol}{blue} \renewcommand{\iniv}{2}
            \else 
                \ifx\nivT\NIV \renewcommand{\nivcol}{green} \renewcommand{\iniv}{3}
                \else
                    \ifx\nivA\NIV \renewcommand{\nivcol}{orange} \renewcommand{\iniv}{4}
                    \else
                        \ifx\nivE\NIV \renewcommand{\nivcol}{red} \renewcommand{\iniv}{5}
                        \else 
                            \ifx\nivC\NIV \renewcommand{\nivcol}{violet} \renewcommand{\iniv}{6}
                            \else fout \fi
                        \fi
                    \fi
                \fi
            \fi 
    \fi
    

    
    
    
    
   \begin{tikzpicture}[scale=0.25]
        % Static part
        \draw[draw=white,fill=violet,thick] (-45:2) -- (-45:3.5) arc(-45:0:3.5) -- (0:2) arc(0:-45:2) -- cycle ;
        \draw[draw=white,fill=red,thick] (45:2cm)-- (45:3.5cm) arc (45:0:3.5) -- (0:2cm) arc (0:45:2);
        \draw[draw=white,fill=orange,thick] (90:2cm)-- (90:3.5cm) arc (90:45:3.5) -- (45:2cm) arc (45:90:2);
        \draw[draw=white,fill=green,thick] (135:2cm)-- (135:3.5cm) arc (135:90:3.5) -- (90:2cm) arc (90:135:2);
        \draw[draw=white,fill=blue,thick] (180:2cm)-- (180:3.5cm) arc (180:135:3.5) -- (135:2cm) arc (135:180:2);
        \draw[draw=white,fill=gray,thick] (225:2cm)-- (225:3.5cm) arc (225:180:3.5) -- (180:2cm) arc (180:225:2);
        \draw[draw=white,fill=\nivcol,thick] ({225-(\iniv)*45}:2cm)-- ({225-(\iniv)*45}:5cm) arc ({225-(\iniv)*45}:{225-(\iniv-1)*45}:5) -- ({225-(\iniv-1)*45}:2cm) arc ({225-(\iniv-1)*45}:{225-(\iniv)*45}:2);
        
        % Speedometer needle
        %\pgfmathsetmacro{\col}{ifthenelse(\NIV==T,"orange",ifthenelse(\NIV==A,"blue"))}
        \node at (0,0) {\textcolor{\nivcol}{\huge{\textbf{\NIV}}}};
        \node at (6,5) [below right,text width=0.80\textwidth] {\OMS}
        \end{tikzpicture} \\
        
    %Aanduidend dat de doelsteling werd gebruikt
    \DTLgetrowindex{\rowidx}{\mydb}{\dtlcolumnindex{\mydb}{REF}}{#1}%
    \dtlgetrow{\mydb}{\rowidx}
    \dtlupdateentryincurrentrow{XIMERA_USE}{Ja}
    \dtlupdateentryincurrentrow{XIMERA_UPDATE}{\datum}
    \dtlrecombine
    \DTLsavedb{\mydb}{LPD_wiskunde2.csv}
  }

\newcommand{\cocalc}{\includegraphics[scale=0.05]{../pictures/cocalc_icon.png}}              %afbeelding cocalc (verwijst naar voorbeelden/oefening in, cocalc

\newcommand{\pcode}[1]{\texttt{\textcolor{red}{#1}}} %python-commando's weergeven in deze stijl

\newcommand{\csv}[2]{\href{#1}{(\includegraphics[scale=0.05]{csv-icon.png} #2)}}

\newcommand{\fystip}{\begin{warning}
	Denk aan!
	\begin{itemize}
		\item correct notatie grootheden.
		\item bij een grootheid hoort een eenheid.
		\item resultaten in wetenschappelijke notatie.
		\item beduidende cijfers.
	\end{itemize}
\end{warning}}



% Instelling om stukjes code weer te geven
    \usepackage{listings}
    
    
    
    %New colors defined below
    \definecolor{codegreen}{rgb}{0,0.6,0}
    \definecolor{codegray}{rgb}{0.5,0.5,0.5}
    \definecolor{codepurple}{rgb}{0.58,0,0.82}
    \definecolor{backcolour}{rgb}{0.95,0.95,0.92}
    
    %Code listing style named "mystyle"
    \lstdefinestyle{mystyle}{
      backgroundcolor=\color{backcolour}, commentstyle=\color{codegreen},
      keywordstyle=\color{red},
      numberstyle=\tiny\color{codegray},
      stringstyle=\color{codepurple},
      basicstyle=\ttfamily\footnotesize,
      breakatwhitespace=false,         
      breaklines=true,                 
      captionpos=b,                    
      keepspaces=true,                 
      numbers=left,                    
      numbersep=5pt,                  
      showspaces=false,                
      showstringspaces=false,
      showtabs=false,                  
      tabsize=2
    }
    
    \lstset{style=mystyle}

%niveau oefeningen

%Specifiek element uit de lijst van kleuren




\newcommand{\oefniv}[1]
{
\ifnum #1=1 
    \def\mycolor{gray}
\else 
    \ifnum #1=2
        \def\mycolor{blue}
    \else
        \ifnum #1=3
            \def\mycolor{green}
        \else
            \ifnum #1=4 
                \def\mycolor{orange}
            \else
                \ifnum #1=5 
                    \def\mycolor{red}
                \else
                    \ifnum #1=6 
                        \def\mycolor{violet}
                    \else
                        \def\mycolor{black}
                    \fi
                \fi 
            \fi 
        \fi
    \fi 
\fi
\begin{tikzpicture}[scale=0.15]
    \draw[draw=white,fill=gray!50,thick] (0, 0) rectangle (2, 2);
    \draw[draw=white,fill=blue!50,thick] (2, 0) rectangle (4, 2);
    \draw[draw=white,fill=green!50,thick] (4, 0) rectangle (6, 2);
    \draw[draw=white,fill=orange!50,thick] (6, 0) rectangle (8, 2);
    \draw[draw=white,fill=red!50,thick] (8, 0) rectangle (10, 2);
    \draw[draw=white,fill=violet!50,thick] (10, 0) rectangle (12, 2);

    \draw[draw=white,fill=\mycolor,thick] ({(#1-1)*2}, 0) -- ({(#1-1)*2},4) -- ({(#1)*2},4) -- ({(#1)*2},0) -- cycle;
\end{tikzpicture}
\\
}

%eigenschappen veer

\usetikzlibrary{decorations.pathmorphing,calc,patterns}

\makeatletter

\def\pgfdecorationspringstraightlinelength{.5cm}
\def\pgfdecorationspringnumberofelement{8}
\def\pgfdecorationspringnaturallength{5cm}
\pgfkeys{%
  /pgf/decoration/.cd,
  spring straight line length/.code={%
    \pgfmathsetlengthmacro\pgfdecorationspringstraightlinelength{#1}},
  spring natural length/.code={%
    \pgfmathsetlengthmacro\pgfdecorationspringnaturallength{#1}},
  spring number of element/.store in=\pgfdecorationspringnumberofelement
}

\pgfdeclaredecoration{coil spring}{straight line}{%
  \state{straight line}[%
    persistent precomputation = {%
      % Compute the effective length of the spring (without the length
      % of the two straight lines): \pgfdecorationspringeffectivelength
      \pgfmathsetlengthmacro{\pgfdecorationspringeffectivelength}%
        {\pgfdecoratedpathlength-2*\pgfdecorationspringstraightlinelength}
      % Compute the effective length of one coil pattern:
      % \pgfdecorationspringeffectivelengthofonecoil
      \pgfmathsetlengthmacro{\pgfdecorationspringeffectivelengthofonecoil}%
        {\pgfdecorationspringeffectivelength/\pgfdecorationspringnumberofelement}
    },
    width = \pgfdecorationspringstraightlinelength,
    next state = draw spring]{%
      \pgfpathlineto{%
        \pgfqpoint{%
          \pgfdecorationspringstraightlinelength}{0pt}}
  }
  \state{draw spring}%
    [width=\pgfdecorationspringeffectivelengthofonecoil,
     repeat state=\pgfdecorationspringnumberofelement-1,next state=final]{%
       \pgfpathcurveto
         {\pgfpoint@onspringcoil{0    }{ 0.555}{1}}
         {\pgfpoint@onspringcoil{0.445}{ 1    }{2}}
         {\pgfpoint@onspringcoil{1    }{ 1    }{3}}
       \pgfpathcurveto
         {\pgfpoint@onspringcoil{1.555}{ 1    }{4}}
         {\pgfpoint@onspringcoil{2    }{ 0.555}{5}}
         {\pgfpoint@onspringcoil{2    }{ 0    }{6}}
       \pgfpathcurveto
         {\pgfpoint@onspringcoil{2    }{-0.555}{7}}
         {\pgfpoint@onspringcoil{1.555}{-1    }{8}}
         {\pgfpoint@onspringcoil{1    }{-1    }{9}}
       \pgfpathcurveto
         {\pgfpoint@onspringcoil{0.445}{-1    }{10}}
         {\pgfpoint@onspringcoil{0    }{-0.555}{11}}
         {\pgfpoint@onspringcoil{0    }{ 0    }{12}}
  }
  \state{final}{%
    \pgfpathlineto{\pgfpointdecoratedpathlast}
  }
}

\def\pgfpoint@onspringcoil#1#2#3{%
  \pgf@x=#1\pgfdecorationsegmentamplitude%
  \pgf@x=.5\pgf@x%
  \pgf@y=#2\pgfdecorationsegmentamplitude%
  \pgfmathparse{0.083333333333*\pgfdecorationspringeffectivelengthofonecoil}%
  \pgf@xa=\pgfmathresult pt
  \advance\pgf@x by#3\pgf@xa%
}


\makeatother

\tikzset{%
  Spring/.style = {%
    decoration = {%
      coil spring,
      spring straight line length = .5cm,
      % To be added
      spring natural length = #1,
      spring number of element = 8,
      amplitude=3mm},
    decorate,
    very thick},
  Spring/.default = {4cm}}


\addPrintStyle{..}





\begin{document}
    \author{Ingmar Herreman}
    \date{Januari 2024}
    \xmtitle{\fys{De energie van een harmonisch trillend systeem}}{}

\doel{F48}    

\begin{center}
\youtube{7FfKaIgArJ8}
\end{center}

Een ongedempte triller is een mooie toepassing van de wet van behoud van “mechanische” energie zoals die in het 4de jaar werd gezien.
\begin{center}
$E_{kin}+E_{pot}=C^{te}$
\end{center}
Een massa die harmonische trilling uitvoert bezit steeds twee vormen van energie, potentiële energie $E_{pot}$ en kinetische energie $E_{kin}$, die samen de mechanische energie $E_{mech}$ van het systeem uitmaken.

\xmsection{De kinetische energie}
\begin{proof}
\begin{align*}
    E_{kin} & = \dfrac{1}{2}m\cdot v^2 \\
    &=\dfrac{1}{2}m\cdot A^2\cdot \omega^2 \cdot \cos^2 (\omega \cdot t+\phi_0)
\end{align*}
Aangezien $\omega=2\pi\cdot f =\sqrt{\dfrac{k}{m}}$ is
\begin{align*}
    E_{kin}&=\dfrac{1}{2}m\cdot A^2\cdot \dfrac{k}{m} \cdot \cos^2 (\omega \cdot t+\phi_0) \\
    &=\dfrac{1}{2}\cdot A^2\cdot k \cdot \cos^2 (\omega \cdot t+\phi_0) 
\end{align*}
\end{proof}


\begin{definition}
    [Kinetische energie]  \ \\
    \begin{center}
        \important{E_{kin}=\dfrac{1}{2}\cdot A^2\cdot k \cdot \cos^2 (\omega \cdot t+\phi_0)}
    \end{center}
\end{definition}
\begin{remark}
    \begin{itemize}
        \item 	De harmonische trillende massa heeft kinetische energie $E_{kin}$ wanneer de snelheid verschillend is van nul.
	\item In het evenwichtspunt is de kinetische energie maximaal.
	\item Bij maximale uitwijking is de kinetische energie gelijk aan 0.
    \end{itemize}
\end{remark}

\xmsection{De potentiële energie}
\begin{proof}
\begin{align*}
    E_{pot} & = \dfrac{1}{2}k\cdot (\Delta x)^2 \\
    &=\dfrac{1}{2}k\cdot A^2\cdot  \sin^2 (\omega \cdot t+\phi_0)
\end{align*}
\end{proof}


\begin{definition}
    [Kinetische energie]  \ \\
    \begin{center}
        \important{E_{pot}=\dfrac{1}{2}k\cdot A^2\cdot  \sin^2 (\omega \cdot t+\phi_0)}
    \end{center}
\end{definition}
\begin{remark}
    \begin{itemize}
        \item
	De harmonische trillende massa heeft potentiële energie $E_{pot}$ wanneer de uitwijking verschillend is van nul.
	\item In het evenwichtspunt is de potentiële energie 0.
	\item Bij maximale uitwijking is de potentiële energie maximaal.

    \end{itemize}
\end{remark}


\xmsection{De mechanische energie}
Volgens de wet van behoud van mechanische energie is de totale hoeveelheid mechanische energie constant.
\begin{proof}
\begin{align*}
    E_{mech} &=E_{kin}+E_{pot} \\
    & =\dfrac{1}{2}\cdot A^2\cdot k \cdot \cos^2 (\omega \cdot t+\phi_0) + \dfrac{1}{2}m\cdot (\Delta x)^2 \\
    &=\dfrac{1}{2}k\cdot A^2  \underbrace{\left(\cos^2 (\omega \cdot t+\phi_0)+\sin^2 (\omega \cdot t+\phi_0)\right)}_{=1} \\
    &=\dfrac{1}{2}k\cdot A^2  \\
    &&f=\dfrac{1}{2\pi}\sqrt{\dfrac{k}{m}} \Leftrightarrow k=4\pi^2\cdot m \cdot f^2 \\
    E_{mech} &= 2\pi^2\cdot m\cdot f^2 \cdot A^2
\end{align*}
\end{proof}


\begin{definition}
    [Mechanische energie]  \ \\
    \begin{center}
        \important{E_{pot}=\dfrac{1}{2}k\cdot A^2=2\pi^2\cdot m\cdot f^2 \cdot A^2}
    \end{center}
\end{definition}

De mechanische energie van een systeem dat een harmonische trilling uitvoert, is afhankelijk van de amplitude. Hoe groter de amplitude, hoe groter de mechanische energie van het systeem.
Bij een ongedempte trilling blijft de amplitude en de mechanische energie van het systeem constant.

\begin{image}
    \begin{tikzpicture}
    \def\tick#1#2{\draw[thick] (#1)++(#2:0.12) --++ (#2-180:0.24)}
  \def\xmax{7}    % max x axis
  \def\ymax{4}    % max y axis
  \def\A{0.8*\xmax} % maximum extension
  \def\E{0.8*\ymax} % maximum total energy
  
  % AXIS
  \draw[->,thick] (0,-0.1*\ymax) -- (0,\ymax) node[left] {$E$};
  \draw[->,thick] (-\xmax,0) -- (\xmax,0) node[right] {$x$};
  \draw[dashed] (-\A,0) --++ (0,0.9*\ymax);
  \draw[dashed] ( \A,0) --++ (0,0.9*\ymax);
  \draw[very thick,blue] (-\A,\E) -- (\A,\E);
  
  % PLOT
  \draw[very thick,teal,samples=100,smooth,variable=\x,domain=-\A:\A]
    plot(\x,{\E/(\A*\A)*\x*\x)});
  \draw[very thick,red,samples=100,smooth,variable=\x,domain=-\A:\A]
    plot(\x,{\E-\E/(\A*\A)*\x*\x)});
  \node[teal,right,fill=white,inner sep=0,scale=0.9] at (0.92*\A,0.7*\E) {$E_{pot} = \frac{1}{2}kx^2$};
  \node[red,right,fill=white,inner sep=0,scale=0.9] at (0.92*\A,0.3*\E) {$E_{kin} = \frac{1}{2}mv^2$};
    \tick{0,\E}{180} node[above right=0,scale=1] {\textcolor{blue}{$E_\text{tot}$}};
  \tick{-\A,0}{90} node[below=0,scale=1] {$-x_\text{max}$};
  \tick{\A,0}{90} node[below=0,scale=1] {$x_\text{max}$};
  
\end{tikzpicture}
\end{image}
\begin{image}
\begin{tikzpicture}
\def\tick#1#2{\draw[thick] (#1)++(#2:0.12) --++ (#2-180:0.24)}
\def\xmax{10} 
  \def\ymax{5} % max y axis
  \def\A{0.8*\ymax} % maximum extension
  \def\om{(2.5*360/(0.94*\xmax))} % angular frequency in degrees
  \def\E{0.8*\ymax}
  
  % AXIS
  \draw[->,thick] (0,-0.1*\ymax) -- (0,\ymax) node[left] {$E$};
  \draw[->,thick] (-0.1*\ymax,0) -- (\xmax,0) node[below] {$t$};
  \draw[very thick,blue] (0,\A) -- (0.95*\xmax,\A);
  
  % PLOT
  \draw[very thick, teal,samples=500,smooth,variable=\x,domain=0:0.94*\xmax]
    plot(\x,{\A*cos(\om*\x)^2});
  \draw[very thick,red,samples=500,smooth,variable=\x,domain=0:0.94*\xmax]
    plot(\x,{\A*sin(\om*\x)^2});
  \tick{0,\A}{0} node[above right=0,scale=1] {\textcolor{blue}{$E_{tot}$}};
   \node[teal,right,fill=white,inner sep=0,scale=0.9] at (0.92*\A+6,0.7*\E) {$E_{pot} = \frac{1}{2}kx^2$};
  \node[red,right,fill=white,inner sep=0,scale=0.9] at (0.92*\A+6,0.3*\E) {$E_{kin} = \frac{1}{2}mv^2$};

  
\end{tikzpicture}

\end{image}

\xmsection{Oefeningen}

\begin{exercise} \oefniv{3}
    Een blokje met massa $150 \ g$ trilt aan een veer met veerconstante $20 \dfrac{N}{m}$ en amplitude $10,0 \ cm$. 
    Bereken de mechanische energie.
\end{exercise}

\begin{exercise} \oefniv{3}
    Een blokje met massa $200 \ g$ wordt aan een veer met lengte $80 \ cm$ gehangen. De veer rekt $20 \ cm$ uit. 
    Het blokje wordt aan het trillen gebracht met een amplitude van $10 \ cm$.
        \begin{question}
	   Bepaal $T$, $f$ en $\omega$
        \end{question}

        \begin{question}
        Bepaal de maximale snelheid en versnelling van het blokje. 
        \end{question}

        \begin{question}
          Bepaal de energie van het trillend blokje.  
        \end{question}
\end{exercise}
\end{document}