\documentclass{ximera}

\input{preamble}
% Extra pakketen



\usepackage{fontawesome5}
\usepackage{xcolor}
\usepackage{physics}
%\usepackage{siunitx}
\usepackage{lipsum}
%\usepackage{circuitikz}
\usepackage{csvsimple}
\usetikzlibrary{3d} % for canvas is
\usetikzlibrary{decorations.pathmorphing,decorations.markings} % for random steps & snake
\usepackage{datatool}
\usepackage{pgffor}
\usepackage{eurosym}
\usepackage{algorithmic}
\usepackage{pythontex} 
\usepackage{xparse}
%Extra commando's

\newcommand{\datum}{\today}
\newcommand{\nivcol}{black}
\newcommand{\iniv}{0} % index van het beheersingsniveau














% In plaats van grijs achtergrond, een rode achtergrond

%\renewcommand{\important}[1]{\ensuremath{\fcolorbox{red}{red!50}{$#1$}}}


% Uitbreidingwet zijn extra's die wel redelijkerwijze tot de leerstof van bv meer geavanceerde versies kunnen behoren voor de wetenschappelijke richtingen

\ifdefined\isXourse
   \ifdefined\xmuitweidingwet
   \else
       \def\xmnouitweidingwet{false}
   \fi
\fi

\ifdefined\xmuitweidingwet
\newcounter{xmuitweidingwet}  % anders error undefined ...  
\includecomment{xmuitweidingwet}
\else
\newtheoremstyle{dotless}{}{}{}{}{}{}{ }{}
\theoremstyle{dotless}
\newtheorem*{xmuitweidingwetnofrills}{}   % nofrills = no accordion; gebruikt dus de dotless theoremstyle!

\newcounter{xmuitweidingwet}
\newenvironment{xmuitweidingwet}[1][ ]%
{% 
	\refstepcounter{xmuitweidingwet}
	\begin{accordion}\begin{accordion-item}[\color{magenta}{{ $\ggg$ UITBREIDING WETENSCHAPPEN}} \arabic{xmuitweidingwet}: #1]%
			\begin{xmuitweidingwetnofrills}%
			}
			{\end{xmuitweidingwetnofrills}\end{accordion-item}\end{accordion}}   
\fi


%uitbreiding wiskunde


\ifdefined\isXourse
   \ifdefined\xmuitweidingwis
   \else
       \def\xmnouitweidingwis{false}
   \fi
\fi

\ifdefined\xmuitweidingwis
\newcounter{xmuitweidingwis}  % anders error undefined ...  
\includecomment{xmuitweidingwis}
\else
\newtheoremstyle{dotless}{}{}{}{}{}{}{ }{}
\theoremstyle{dotless}
\newtheorem*{xmuitweidingwisnofrills}{}   % nofrills = no accordion; gebruikt dus de dotless theoremstyle!

\newcounter{xmuitweidingwis}
\newenvironment{xmuitweidingwis}[1][ ]%
{% 
	\refstepcounter{xmuitweidingwis}
	\begin{accordion}\begin{accordion-item}[\color{purple}{{$\infty$ UITBREIDING WISKUNDE}} \arabic{xmuitweidingwis}: #1]%
			\begin{xmuitweidingwisnofrills}%
			}
			{\end{xmuitweidingwisnofrills}\end{accordion-item}\end{accordion}}   
\fi


% Oefening enkel voor wetenschappen en wiskunde


\newenvironment{oefwet}[1]
{   \textcolor{teal}{
    $\ggg WETENSCHAPPEN$ \hrulefill \\
        	#1 
        	}
}

% % afbeelding vak
\newcommand{\wis}[1]{$\infty$ #1}
\newcommand{\wet}[1]{ #1}
\newcommand{\fys}[1]{$\nabla$ #1}
\newcommand{\info}[1]{$\vdots$ #1}

\newcommand{\arch}[1]{$\measuredangle
$ #1}

\newcommand{\eco}[1]{\officialeuro #1}

%tabel grootheid en eenheid
% \newenvironment{eenheid}[4]
% {
% \renewcommand{\arraystretch}{2}

%     \begin{table}[h]
%         \centering
%         \begin{tabular}{|c|c||c|c|}
%         \hline
%              \large{Grootheid} &  \large{Symbool} & \large{Eenheid} & 
%              \large{Symbool} \\
%              \hline \hline
%              \cellcolor{teal!50}\textbf{\LARGE{#1}} & \cellcolor{teal!50}\textbf{\LARGE{#2}} & \cellcolor{teal!50}\textbf{\LARGE{#3}} & \cellcolor{teal!50}\textbf{\LARGE{#4}} \\
%              \hline
%         \end{tabular}
%     \end{table}

% \renewcommand{\arraystretch}{1}
% }
%Leerplaandoelstelling bepalen
\newcommand{\doel}[1]
  {
    \DTLsetseparator{;}  %csv-scheidingsteken
    
    \def\mydb{doel}
    \edef\mydb{\mydb{}#1}
    \DTLloaddb{\mydb}{../Data/LPD_wiskunde.csv} % laad csv-bestand 
    \DTLassignfirstmatch{\mydb}{REF}{#1} %lees een rij met kolommen
    {\REF=REF,\LPD=LPD,\OND=ONDERWERP,\OMS=OMSCHRIJVING,\NIV=BEHEERSINGSNIVEAU}

    \def\nivM{M} %memoriseren index 1
    \def\nivB{B} %begrijpen index 2
    \def\nivT{T} %toepassen index 3
    \def\nivA{A} %analyseren index 4
    \def\nivE{E} %evalueren index 5
    \def\nivC{C} %creeren index 6
    \ifx\nivM\NIV \renewcommand{\nivcol}{gray} \renewcommand{\iniv}{1}
    \else
            \ifx\nivB\NIV \renewcommand{\nivcol}{blue} \renewcommand{\iniv}{2}
            \else 
                \ifx\nivT\NIV \renewcommand{\nivcol}{green} \renewcommand{\iniv}{3}
                \else
                    \ifx\nivA\NIV \renewcommand{\nivcol}{orange} \renewcommand{\iniv}{4}
                    \else
                        \ifx\nivE\NIV \renewcommand{\nivcol}{red} \renewcommand{\iniv}{5}
                        \else 
                            \ifx\nivC\NIV \renewcommand{\nivcol}{violet} \renewcommand{\iniv}{6}
                            \else fout \fi
                        \fi
                    \fi
                \fi
            \fi 
    \fi
    

    
    
    
    
   \begin{tikzpicture}[scale=0.25]
        % Static part
        \draw[draw=white,fill=violet,thick] (-45:2) -- (-45:3.5) arc(-45:0:3.5) -- (0:2) arc(0:-45:2) -- cycle ;
        \draw[draw=white,fill=red,thick] (45:2cm)-- (45:3.5cm) arc (45:0:3.5) -- (0:2cm) arc (0:45:2);
        \draw[draw=white,fill=orange,thick] (90:2cm)-- (90:3.5cm) arc (90:45:3.5) -- (45:2cm) arc (45:90:2);
        \draw[draw=white,fill=green,thick] (135:2cm)-- (135:3.5cm) arc (135:90:3.5) -- (90:2cm) arc (90:135:2);
        \draw[draw=white,fill=blue,thick] (180:2cm)-- (180:3.5cm) arc (180:135:3.5) -- (135:2cm) arc (135:180:2);
        \draw[draw=white,fill=gray,thick] (225:2cm)-- (225:3.5cm) arc (225:180:3.5) -- (180:2cm) arc (180:225:2);
        \draw[draw=white,fill=\nivcol,thick] ({225-(\iniv)*45}:2cm)-- ({225-(\iniv)*45}:5cm) arc ({225-(\iniv)*45}:{225-(\iniv-1)*45}:5) -- ({225-(\iniv-1)*45}:2cm) arc ({225-(\iniv-1)*45}:{225-(\iniv)*45}:2);
        
        % Speedometer needle
        %\pgfmathsetmacro{\col}{ifthenelse(\NIV==T,"orange",ifthenelse(\NIV==A,"blue"))}
        \node at (0,0) {\textcolor{\nivcol}{\huge{\textbf{\NIV}}}};
        \node at (6,5) [below right,text width=0.80\textwidth] {\OMS}
        \end{tikzpicture} \\
        
    %Aanduidend dat de doelsteling werd gebruikt
    \DTLgetrowindex{\rowidx}{\mydb}{\dtlcolumnindex{\mydb}{REF}}{#1}%
    \dtlgetrow{\mydb}{\rowidx}
    \dtlupdateentryincurrentrow{XIMERA_USE}{Ja}
    \dtlupdateentryincurrentrow{XIMERA_UPDATE}{\datum}
    \dtlrecombine
    \DTLsavedb{\mydb}{LPD_wiskunde2.csv}
  }

\newcommand{\cocalc}{\includegraphics[scale=0.05]{../pictures/cocalc_icon.png}}              %afbeelding cocalc (verwijst naar voorbeelden/oefening in, cocalc

\newcommand{\pcode}[1]{\texttt{\textcolor{red}{#1}}} %python-commando's weergeven in deze stijl

\newcommand{\csv}[2]{\href{#1}{(\includegraphics[scale=0.05]{csv-icon.png} #2)}}

\newcommand{\fystip}{\begin{warning}
	Denk aan!
	\begin{itemize}
		\item correct notatie grootheden.
		\item bij een grootheid hoort een eenheid.
		\item resultaten in wetenschappelijke notatie.
		\item beduidende cijfers.
	\end{itemize}
\end{warning}}



% Instelling om stukjes code weer te geven
    \usepackage{listings}
    
    
    
    %New colors defined below
    \definecolor{codegreen}{rgb}{0,0.6,0}
    \definecolor{codegray}{rgb}{0.5,0.5,0.5}
    \definecolor{codepurple}{rgb}{0.58,0,0.82}
    \definecolor{backcolour}{rgb}{0.95,0.95,0.92}
    
    %Code listing style named "mystyle"
    \lstdefinestyle{mystyle}{
      backgroundcolor=\color{backcolour}, commentstyle=\color{codegreen},
      keywordstyle=\color{red},
      numberstyle=\tiny\color{codegray},
      stringstyle=\color{codepurple},
      basicstyle=\ttfamily\footnotesize,
      breakatwhitespace=false,         
      breaklines=true,                 
      captionpos=b,                    
      keepspaces=true,                 
      numbers=left,                    
      numbersep=5pt,                  
      showspaces=false,                
      showstringspaces=false,
      showtabs=false,                  
      tabsize=2
    }
    
    \lstset{style=mystyle}

%niveau oefeningen

%Specifiek element uit de lijst van kleuren




\newcommand{\oefniv}[1]
{
\ifnum #1=1 
    \def\mycolor{gray}
\else 
    \ifnum #1=2
        \def\mycolor{blue}
    \else
        \ifnum #1=3
            \def\mycolor{green}
        \else
            \ifnum #1=4 
                \def\mycolor{orange}
            \else
                \ifnum #1=5 
                    \def\mycolor{red}
                \else
                    \ifnum #1=6 
                        \def\mycolor{violet}
                    \else
                        \def\mycolor{black}
                    \fi
                \fi 
            \fi 
        \fi
    \fi 
\fi
\begin{tikzpicture}[scale=0.15]
    \draw[draw=white,fill=gray!50,thick] (0, 0) rectangle (2, 2);
    \draw[draw=white,fill=blue!50,thick] (2, 0) rectangle (4, 2);
    \draw[draw=white,fill=green!50,thick] (4, 0) rectangle (6, 2);
    \draw[draw=white,fill=orange!50,thick] (6, 0) rectangle (8, 2);
    \draw[draw=white,fill=red!50,thick] (8, 0) rectangle (10, 2);
    \draw[draw=white,fill=violet!50,thick] (10, 0) rectangle (12, 2);

    \draw[draw=white,fill=\mycolor,thick] ({(#1-1)*2}, 0) -- ({(#1-1)*2},4) -- ({(#1)*2},4) -- ({(#1)*2},0) -- cycle;
\end{tikzpicture}
\\
}

%eigenschappen veer

\usetikzlibrary{decorations.pathmorphing,calc,patterns}

\makeatletter

\def\pgfdecorationspringstraightlinelength{.5cm}
\def\pgfdecorationspringnumberofelement{8}
\def\pgfdecorationspringnaturallength{5cm}
\pgfkeys{%
  /pgf/decoration/.cd,
  spring straight line length/.code={%
    \pgfmathsetlengthmacro\pgfdecorationspringstraightlinelength{#1}},
  spring natural length/.code={%
    \pgfmathsetlengthmacro\pgfdecorationspringnaturallength{#1}},
  spring number of element/.store in=\pgfdecorationspringnumberofelement
}

\pgfdeclaredecoration{coil spring}{straight line}{%
  \state{straight line}[%
    persistent precomputation = {%
      % Compute the effective length of the spring (without the length
      % of the two straight lines): \pgfdecorationspringeffectivelength
      \pgfmathsetlengthmacro{\pgfdecorationspringeffectivelength}%
        {\pgfdecoratedpathlength-2*\pgfdecorationspringstraightlinelength}
      % Compute the effective length of one coil pattern:
      % \pgfdecorationspringeffectivelengthofonecoil
      \pgfmathsetlengthmacro{\pgfdecorationspringeffectivelengthofonecoil}%
        {\pgfdecorationspringeffectivelength/\pgfdecorationspringnumberofelement}
    },
    width = \pgfdecorationspringstraightlinelength,
    next state = draw spring]{%
      \pgfpathlineto{%
        \pgfqpoint{%
          \pgfdecorationspringstraightlinelength}{0pt}}
  }
  \state{draw spring}%
    [width=\pgfdecorationspringeffectivelengthofonecoil,
     repeat state=\pgfdecorationspringnumberofelement-1,next state=final]{%
       \pgfpathcurveto
         {\pgfpoint@onspringcoil{0    }{ 0.555}{1}}
         {\pgfpoint@onspringcoil{0.445}{ 1    }{2}}
         {\pgfpoint@onspringcoil{1    }{ 1    }{3}}
       \pgfpathcurveto
         {\pgfpoint@onspringcoil{1.555}{ 1    }{4}}
         {\pgfpoint@onspringcoil{2    }{ 0.555}{5}}
         {\pgfpoint@onspringcoil{2    }{ 0    }{6}}
       \pgfpathcurveto
         {\pgfpoint@onspringcoil{2    }{-0.555}{7}}
         {\pgfpoint@onspringcoil{1.555}{-1    }{8}}
         {\pgfpoint@onspringcoil{1    }{-1    }{9}}
       \pgfpathcurveto
         {\pgfpoint@onspringcoil{0.445}{-1    }{10}}
         {\pgfpoint@onspringcoil{0    }{-0.555}{11}}
         {\pgfpoint@onspringcoil{0    }{ 0    }{12}}
  }
  \state{final}{%
    \pgfpathlineto{\pgfpointdecoratedpathlast}
  }
}

\def\pgfpoint@onspringcoil#1#2#3{%
  \pgf@x=#1\pgfdecorationsegmentamplitude%
  \pgf@x=.5\pgf@x%
  \pgf@y=#2\pgfdecorationsegmentamplitude%
  \pgfmathparse{0.083333333333*\pgfdecorationspringeffectivelengthofonecoil}%
  \pgf@xa=\pgfmathresult pt
  \advance\pgf@x by#3\pgf@xa%
}


\makeatother

\tikzset{%
  Spring/.style = {%
    decoration = {%
      coil spring,
      spring straight line length = .5cm,
      % To be added
      spring natural length = #1,
      spring number of element = 8,
      amplitude=3mm},
    decorate,
    very thick},
  Spring/.default = {4cm}}


\addPrintStyle{..}





\begin{document}
    \author{Ingmar Herreman}
    \date{Februari 2024}
    \xmtitle{\fys{De snelheid en versnelling bij een harmonische trilling}}{}

    \doel{F48}

\xmsection{De snelheidsvergelijking $v(t)$}

Een harmonische trillend voorwerp gaat op en neer rond het evenwichtspunt en heeft daarbij op elk $t$ een andere uitwijking $x(t)$, maar ook een andere snelheid $v(t)$.

We bekijken de harmonische trilling:
\begin{center}
    $x(t)=A\cdot \sin(\omega \cdot t + \varphi_0)$
\end{center}
Uit de kinematica weten we dat de snelheid $v$ op het ogenblik $t$ gegeven wordt door:
\begin{align*}
    v(t)&=\dfrac{dx(t)}{dt} \\
    &=\dfrac{d(A\cdot\sin(\omega\cdot t + \varphi_0))}{dt} \\
    &=A\cdot \cos(\omega \cdot t + \varphi_0) \cdot \omega \\
    &=A\cdot \omega \cos(\omega \cdot t + \varphi_0) 
        \end{align*}

We noemen dit \textbf{de snelheidsvergelijking} van een harmonische trilling. Hiermee kan je op elk tijdstip de snelheid van de triller berekenen.

\begin{definition}
[De snelheidsvergelijking] \ \\
\begin{center}
    \important{v(t)=A\cdot \omega \cdot \cos(\omega \cdot t + \varphi_0)}
\end{center}
\end{definition}
    \begin{image}
        \begin{tikzpicture}
        
    \begin{axis}[restrict y to domain=-20:1000,
                ylabel={$x(m)$},
                xlabel={$t(s)$},
                %axis equal image=true,
                grid,
                grid style={gray!50},
                grid=both,
            axis y line=center,
            axis x line=middle, 
            axis on top=true,
            xmin=-1,
            xmax=10,
            ymin=-1.1,
            ymax=1.1,
            ytick={\empty},
            xtick={\empty},
            x=1cm,y=2 cm]
        
        \addplot[blue,ultra thick,samples=500,smooth,domain=-5:10] {sin(deg(x))}; 
        \draw[black,dashed] (axis cs:2*pi,1) -- (axis cs:2*pi,-1.2);
        \draw[black,dashed] (axis cs:pi/2,1) -- (axis cs:pi/2,-1.2);
        \draw[blue,-latex] (axis cs:10*pi/4,0) -- (axis cs:10*pi/4,1) node[midway,right]{$A$};
        
        
    \end{axis}
    \end{tikzpicture}
\end{image}

\begin{image}
    \begin{tikzpicture}
            
    \begin{axis}[restrict y to domain=-20:1000,
                ylabel={$v\left(\frac{m}{s}\right)$},
                xlabel={$t(s)$},
                %axis equal image=true,
                grid,
                grid style={gray!50},
                grid=both,
            axis y line=center,
            axis x line=middle, 
            axis on top=true,
            xmin=-1,
            xmax=10,
            ymin=-1.1,
            ymax=1.1,
            ytick={\empty},
            xtick={\empty},
            x=1cm,y=2 cm]
        
        \addplot[teal,ultra thick,samples=500,smooth,domain=-5:10] {cos(deg(x))}; 
        \draw[black,dashed] (axis cs:2*pi,1.2) -- (axis cs:2*pi,-1);
        \draw[black,dashed] (axis cs:pi/2,1.2) -- (axis cs:pi/2,-1);
        \draw[teal,-latex] (axis cs:8*pi/4,0) -- (axis cs:8*pi/4,1) node[midway,right]{$A\cdot \omega$};
        
        
    \end{axis}
    \end{tikzpicture}
    \end{image}

\begin{itemize}
    \item De snelheid is positief als de triller naar boven beweegt en negatief wanneer hij naar onder beweegt.
    \item De snelheid van de harmonische triller is nul in de uiterste punten, dus als de uitwijking $x(t)=-A$ of $x(t)=+A$. 
    Dat komt omdat het voorwerp daar even stil staat om vervolgens terug te keren en in de tegengestelde zin te bewegen.
    \item De grootte van de snelheid is maximaal als de triller door het evenwichtspunt passeert.

\end{itemize}

\xmsection{De versnellingsvergelijking $a(t)$}

Aangezien de snelheid van een harmonische trillend voorwerp op elk tijdstip $t$ verschillend is, moet de triller een versnelling $a$ hebben. Ook deze versnelling is niet constant en hangt dus af van de tijd.

We weten reeds dat de snelheid $v(t)$ van een harmonische trilling gegeven wordt door:
\begin{center}
    $v(t)=A\cdot \omega \cdot \cos(\omega \cdot t + \varphi_0)$
\end{center}
Uit de kinematica weten we dat de versnelling $a$ op het ogenblik $t$ gegeven wordt door:
\begin{align*}
    a(t)&=\dfrac{dv(t)}{dt} \\
        &=\dfrac{d(A\cdot \omega \cdot \cos(\omega \cdot t + \varphi_0))}{dt} \\
          &=-A\cdot \omega \sin(\omega \cdot t + \varphi_0)\cdot \omega \\
          &=-A\cdot \omega^2 \sin(\omega \cdot t + \varphi_0) \\
          &=-\omega^2 \cdot x(t)
\end{align*}

We noemen dit \textbf{de versnellingsvergelijking} van een harmonische trilling. Hiermee kan je op elk tijdstip de snelheid van de triller berekenen.

\begin{definition}
[De versnellingsvergelijking] \ \\
\begin{center}
    \important{a(t)=-A\cdot \omega^2\cdot \sin(\omega \cdot t + \varphi_0)=-\omega^2 \cdot x(t)}
\end{center}
\end{definition}
    \begin{image}
        \begin{tikzpicture}
        
    \begin{axis}[restrict y to domain=-20:1000,
                ylabel={$x(m)$},
                xlabel={$t(s)$},
                %axis equal image=true,
                grid,
                grid style={gray!50},
                grid=both,
            axis y line=center,
            axis x line=middle, 
            axis on top=true,
            xmin=-1,
            xmax=10,
            ymin=-1.1,
            ymax=1.1,
            ytick={\empty},
            xtick={\empty},
            x=1cm,y=2 cm]
        
        \addplot[blue,ultra thick,samples=500,smooth,domain=-5:10] {sin(deg(x))}; 
        \draw[black,dashed] (axis cs:2*pi,1) -- (axis cs:2*pi,-1.2);
        \draw[black,dashed] (axis cs:pi/2,1) -- (axis cs:pi/2,-1.2);
        \draw[blue,-latex] (axis cs:10*pi/4,0) -- (axis cs:10*pi/4,1) node[midway,right]{$A$};
        
        
    \end{axis}
    \end{tikzpicture}
\end{image}

\begin{image}
\begin{tikzpicture}
        
\begin{axis}[restrict y to domain=-20:1000,
            ylabel={$a\left(\frac{m}{s^2}\right)$},
            xlabel={$t(s)$},
            %axis equal image=true,
            grid,
            grid style={gray!50},
            grid=both,
        axis y line=center,
        axis x line=middle, 
        axis on top=true,
        xmin=-1,
        xmax=10,
        ymin=-1.1,
        ymax=1.1,
        ytick={\empty},
        xtick={\empty},
        x=1cm,y=2 cm]
    
     \addplot[red,ultra thick,samples=500,smooth,domain=-5:10] {-sin(deg(x))}; 
     \draw[black,dashed] (axis cs:2*pi,1.2) -- (axis cs:2*pi,-1);
     \draw[black,dashed] (axis cs:pi/2,1.2) -- (axis cs:pi/2,-1);
     \draw[red,-latex] (axis cs:6*pi/4,0) -- (axis cs:6*pi/4,1) node[midway,right]{$A \omega^2$};
     
    
\end{axis}
\end{tikzpicture}
\end{image}

\begin{itemize}
    \item De versnelling van de harmonische triller is nul als de snelheid maximaal is. De triller bevindt zich dan in het evenwichtspunt.
    \item De grootte van de versnelling is maximaal als de snelheid nul is. De triller bevindt zich dan in de uiterste standen.
\end{itemize}


\begin{image}
    \includegraphics{Fysica/Trillingen/HT_snelheid_versnelling}
\end{image}

\begin{image}
    
\begin{tikzpicture}
        
\begin{axis}[restrict y to domain=-20:1000,
            ylabel={},
            xlabel={$t(s)$},
            %axis equal image=true,
            grid,
            grid style={gray!50},
            grid=both,
        axis y line=center,
        axis x line=middle, 
        axis on top=true,
        xmin=-1,
        xmax=5,
        ymin=-5,
        ymax=5,
        ytick={\empty},
        xtick={\empty},
        x=2cm,y=1 cm]

         \draw[gray,thin] (axis cs: -5,-4) -- (axis cs: 5,-4);
        \draw[gray,thin] (axis cs: -5,-3) -- (axis cs: 5,-3);
    \draw[gray,thin] (axis cs: -5,-2) -- (axis cs: 5,-2);
    \draw[gray,thin] (axis cs: -5,-1) -- (axis cs: 5,-1);
    \draw[gray,thin] (axis cs: -5,1) -- (axis cs: 5,1);
    \draw[gray,thin] (axis cs: -5,2) -- (axis cs: 5,2);
    \draw[gray,thin] (axis cs: -5,3) -- (axis cs: 5,3);
    \draw[gray,thin] (axis cs: -5,4) -- (axis cs: 5,4);

    \draw[gray,thin] (axis cs: pi/4,-5) -- (axis cs: pi/4,5);
    \draw[gray,thin] (axis cs: 2*pi/4,-5) -- (axis cs: 2*pi/4,5);
    \draw[gray,thin] (axis cs: 3*pi/4,-5) -- (axis cs: 3*pi/4,5);
    \draw[gray,thin] (axis cs: 4*pi/4,-5) -- (axis cs: 4*pi/4,5);
    \draw[gray,thin] (axis cs: 5*pi/4,-5) -- (axis cs: 5*pi/4,5);
    \draw[gray,thin] (axis cs: 6*pi/4,-5) -- (axis cs: 6*pi/4,5);
    \draw[gray,thin] (axis cs: -pi/4,-5) -- (axis cs: -pi/4,5);
    
     \addplot[blue,ultra thick,samples=500,smooth,domain=-5:10] {sin(2*deg(x))}; 
     \addplot[teal,ultra thick,samples=500,smooth,domain=-5:10] {2*cos(2*deg(x))}; 
     \addplot[red,ultra thick,samples=500,smooth,domain=-5:10] {-4*sin(2*deg(x))}; 
     \legend{$x(t)$,$v(t)$,$a(t)$};
\end{axis}
\end{tikzpicture}
\end{image}

\xmsection{Oefeningen}

\begin{exercise} \oefniv{3}  
	Een trilling wordt beschreven door de volgende vergelijking: \\
$x(t)=0,50\ m \cdot \sin(20,0\cdot \ \frac{1}{s}\cdot t+1)$

\begin{question}
    Maak in Geogebra de trillingsgrafiek. 
\end{question}

\begin{question}
    Bepaal grafisch op de positie, de snelheid en de versnelling op $t=0\ s$ en $t=0,10 \ s$.
    \begin{oplossing} \ \\
        \begin{itemize}
            \item   $x(0)=4,2\cdot 10^{-1} \ m$  \\
                    $x(10)=-3,0\cdot 10^{-2} \ m$
            \item   $v(0)=5,4\ \frac{m}{s}$  \\
                    $v(10)=1,0\cdot 10^{1} \ \frac{m}{s}$
            \item   $a(0)=-1,7\cdot 10^{3} \frac{m}{s^2}$  \\
                    $a(10)=1,2\cdot 10^{1} \ \frac{m}{s^2}$
        \end{itemize}
    \end{oplossing}
\end{question}

\begin{question}
    Bepaal wiskundig op de positie, de snelheid en de versnelling op $t=0\ s$ en $t=0,10 \ s$.
    \begin{oplossing} \ \\
        \begin{itemize}
            \item   $x(0)=4,2\cdot 10^{-1} \ m$  \\
                    $x(10)=-3,0\cdot 10^{-2} \ m$
            \item   $v(t)=\dfrac{d(x(t))}{dt}=10\cdot \cos(20x+1)$ \\
                    $v(0)=5,4\ \frac{m}{s}$  \\
                    $v(10)=1,0\cdot 10^{1} \ \frac{m}{s}$
            \item   $a(t)=\dfrac{d(v(t))}{dt}=-200\cdot \sin(20x+1)$
                    $a(0)=-1,7\cdot 10^{3} \frac{m}{s^2}$  \\
                    $a(10)=1,2\cdot 10^{1} \ \frac{m}{s^2}$
        \end{itemize}
    \end{oplossing}

\end{question}

\end{exercise}

\begin{exercise} \oefniv{3}
    Een systeem voert een harmonische trilling uit beschreven door de volgende trillingsvergelijking:  \\
    $x(t)=0,250\cdot\sin(380\cdot t + \frac{\pi}{2})$
    
    \begin{question}
        Stel de snelheidsvergelijking op.
        \begin{oplossing}
            $v(t)=\dfrac{x(t)}{dt}=-95,0 \cdot\cos(380\cdot t + \frac{\pi}{2})$
        \end{oplossing}
    \end{question}

    \begin{question}
        Hoe groot is de maximale snelheid?
        \begin{oplossing}
            Aangezien $-1\leq\cos x\leq 1$ is de maximale snelheid $v_{max}=9,50 \cdot 10^{1} \ \dfrac{m}{s}$
        \end{oplossing}
    \end{question}

    \begin{question}
        Bereken de uitwijking op $t=0,00 \ s$
        \begin{oplossing}
            $x(0)=2,50\cdot 10^{-1} \ m$
        \end{oplossing}
    \end{question}

    \begin{question}
        Wanneer is de snelheid gelijk aan nul?
        \begin{oplossing}
            \begin{align*}
                &v(t)=0 \\
                \Leftrightarrow & -95,0 \cdot\cos(380\cdot t + \frac{\pi}{2})=0 \\
                \Leftrightarrow & \cos(380\cdot t + \frac{\pi}{2})=0 \\
                \Leftrightarrow & 380\cdot t + \frac{\pi}{2}=\frac{\pi}{2}+k\pi & (k \in \mathbb{R}) \\
                \Leftrightarrow & 380\cdot t =k\pi  \\
                \Leftrightarrow & t =\dfrac{k\pi}{380} 
            \end{align*}
        \end{oplossing}
    \end{question}
\end{exercise}

\begin{exercise} \oefniv{3}
    Het eindpunt van een been van een stemvork trilt harmonisch met frequentie $440 \ Hz$ en amplitude $0,10 \ mm$.
    \begin{question}
        Stel de trillingsvergelijking op voor dat punt (stel de beginfase =$0$).
        $\answer[onlinenoinput]{x(t)=0,10 \ mm \cdot \sin(880  \pi \cdot t)}$
        \begin{oplossing}
            $x(t)=A\cdot \sin(\omega \cdot t)$
            \begin{itemize}
                \item $A=0,10 \ mm$
                \item $\omega=2\pi \cdot f=2\pi \cdot 440=880 \pi$ 
            \end{itemize}
            $x(t)=0,10 \ mm \cdot \sin(880  \pi \cdot t)$
        \end{oplossing}
    \end{question}

    \begin{question}
        Bereken de maximale snelheid en maximale versnelling voor het eindpunt.
        \begin{oplossing}
            \begin{itemize}
            \item $v_{max}=2,8 \cdot 10^{3}  \frac{mm}{s}=2,8  \frac{m}{s}$
            \item $a_{max}=7,6 \cdot 10^{5}  \frac{mm}{s^2}=7,6\cdot 10^2  \frac{m}{s^2}$
            \end{itemize}
        \end{oplossing}
    \end{question}
\end{exercise}

\begin{exercise} \oefniv{3}
    Het eindpunt van de vleugel van een insect beweegt harmonisch met frequentie $65 \ Hz$ en amplitude $3,0 \ mm$. 
    \begin{question}
    Bepaal de maximale snelheid en versnelling van dat punt.
    \begin{oplossing}
        $x(t)=A\cdot \sin(\omega \cdot t)$
            \begin{itemize}
                \item $A=3,0 \ mm$
                \item $\omega=2\pi \cdot f=2\pi \cdot 65=130 \pi$ 
            \end{itemize}
            $x(t)=3,0 \ mm \cdot \sin(130  \pi \cdot t)$
        \begin{itemize}
        \item $v_{max}=1,2 \cdot 10^{3}  \frac{mm}{s}=1,2  \frac{m}{s}$
        \item $a_{max}=5,0 \cdot 10^{5}  \frac{mm}{s^2}=5,0\cdot 10^2  \frac{m}{s^2}$
        \end{itemize}
    \end{oplossing}
    \end{question}


\end{exercise}
\end{document}