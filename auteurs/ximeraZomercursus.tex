\documentclass{ximera}
\input{../preamble.tex}
\addPrintStyle{..}

\begin{document}\label{act:ximeraArchitectuur}
    \author{Wim Obbels}
	\xmtitle{Ximera Zomercursus}{Afspraken en handleiding voor auteurs of editors van deze Zomercursus modules.}

%\subsection{Zomercursus met Ximera}

\subsection{Documentclasses ximera en xourse}
%\TeX nische info: \link{https://github.com/XimeraProject/ximeraLatex}


Ximera structureert leereenheden in 'activiteiten' (\verb|activity|): op-zichzelf-staande \LaTeX~bestanden met documentclass \verb|ximera| die telkens één webpagina vormen en één of meerdere welomschreven leerdoelen (\verb|outcomes| in het Ximerees) trachten te bereiken. 
Voorbeelden kunnen zijn: \verb|breuken.tex| behandelt de definitie en basiseigenschappen van breuken (inclusief enkele voorbeelden en oefeningetjes, want anders kan je dat niet leren), terwijl \verb|breuken_oefeningen.tex| extra oefeningen bevat, \verb|breuken_uitdagendeoefeningen.tex| bijvoorbeeld enkele niet triviale problemen en \verb|breuken_van_veeltermen.tex| extra theorie geeft.
Per conventie zitten de oefeningen in een subfolder \verb|exercises|.

Elke \verb|\documentclass{ximera}| \textit{bevat} dus een dergelijke 'activiteit': een duidelijk afgelijnd en in principe op-zichzelf-staand stukje leerstof. 

Een \verb|\documentclass{xourse}| daarentegen is een \textit{cursus} die dergelijke activiteiten of oefeningen \textit{omvat}, en bestaat uit een lijst \verb|ximera|'s: een \verb|xourse| 'include' een aantal activiteiten, en eventueel een specifieke inleiding of voorwoord, en is dus een bijzonder eenvoudig tex-document. Het is bijgevolg ook erg eenvoudig om op basis van bestaande activities nieuwe xourses samen te stellen.

De activiteiten zijn ondergebracht in folders per onderwerp: basisrekenvaardigheden, limieten, functies, afgeleiden,\ldots.  Met deze activiteiten zijn onder meer volgende xourses samengesteld:
\begin{enumerate}
	\item  ZomercursusZ.tex  (het A-programma voor zelfstudie)
   	\item  ZomercursusB.tex/ZomercursusI.tex  (versies voor B-programma en Informatici)
   	\item  lesweekversieA.tex (versie voor het A-programma 2021)
\end{enumerate}
Volgende xourses zijn niet bedoeld voor eindgebruikers (maar wel bereikbaar voor wie de url's kent):
\begin{enumerate}
   	\item  CatalogusZ.tex (een lijst van alle afgewerkte activities, waaruit kan gekozen worden om xourses samen te stellen)
	\item  Folder ZomercursusA: een xourse per dag voor de (begeleide) Zomercursusweek 2020
    \item  De meeste folders bevatten een xourse.tex die alle activities bevat uit die folder. Deze xourses zijn bedoeld voor auteurs/ontwikkelaars, en niet voor eindgebruikers. In het bijzonder bevatten ze activities die nog niet afgewerkt zijn.
\end{enumerate}


Op \url{https://set.kuleuven.be/voorkennis/zomercursus} is er een automatisch overzicht van alle xourses in het zomercursus-wiskunde repo (ttz: alle \verb|xourse|'s \textit{met een titel}; xourses zonder titel worden niet opgenomen. Dit overzicht is niet publiek raadpleegbaar (je moet het private paswoord kennen).

\begin{xmuitweiding}[Over de status van de Zomercursus 2019]
	
De Modules van de Zomercursus 2019 werden grotendeels omgezet naar ximera-activiteiten (met  minimale aanpassingen, in folders Mxx), en gebundeld in 2 xourses, namelijk ZomercursusA.tex en ZomercursusB.tex, die elk een lijstje van modules bevatten. Ze zijn gearchiveerd onder \href{https://gitlab.kuleuven.be/monitoraat-wet/zomercursus-wiskunde/-/tags?search=archief}{twee tags}...  
Die zomercursussen lijken dus zeer erg op de editie 2019. Voor edities 2020 en later dient te worden beslist of deze modules minimaal verder worden aangepast (layout, figuren), dan wel of de zomercursus 2020 (grotendeels) uit de nieuw ontwikkelde en/of grondig aangepaste modules opnieuw wordt samengesteld.
\end{xmuitweiding}


\subsubsection{Environments}

Ximera definieert out-of-the-box een beperkt aantal environments die zowel in PDF als online een gepaste layout of functionaliteit bieden:
\begin{enumerate}
    \item theorem-like omgevingen: \verb|definition|, \verb|notation|, \verb|axiom|, \verb|proposition|, \verb|theorem|, \verb|proof|, \verb|explanation|, \verb|example|, \verb|remark|, \verb|warning|, \verb|observation|, \verb|algorithm|, \verb|claim|, \verb|conclusion|, \verb|condition|, \verb|conjecture|, \verb|corollary|, \verb|criterion|, \verb|fact|, \verb|lemma|, \verb|formula|, \verb|idea|, \verb|model|, \verb|paradox|, \verb|prodecure|, \verb|summary|, \verb|template|
    
    Deze omgevingen bepalen, net zoals in standaard \LaTeX, de header, de layout en de nummering van verschillende onderdelen van de tekst. En voor deze omgevingen voorziet Ximera bovendien een HTML styling.
    
    Het gebruik is zoals gebruikelijk in \LaTeX:
    \textbackslash \verb|begin{definition}[titel] INHOUD |\textbackslash \verb|end{definition}|
    
    waarbij tussen vierkante haakjes optioneel een titel kan worden meegegeven.
    
    \item problem-like omgevingen: \verb|problem|, \verb|exercise|, \verb|exploration|, \verb|question|
    
    Deze omgevingen functioneren zoals de bovenstaande, maar ze kunnen bovendien commando's en andere environments bevatten die online interactieve onderdelen genereren waarop \textit{antwoorden} kunnen worden ingegeven. Die antwoorden worden naar de Ximera-server gestuurd, waardoor eventueel rapporten en analyses kunnen worden gemaakt. Deze laatste functionaliteit is via LTI integreerbaar met bv Toledo.
    
    In de PDF verschijnen ofwel al direct de juiste antwoorden, ofwel keuzelijstjes of invulpuntjes (afhankelijk van opties die bij het compilen worden op- of afgezet).
    
    Gebruik \verb|example| voor voorbeelden, en \verb|exercise| voor oefeningen.
    De omgevingen  \verb|problem| en \verb|exploration| worden niet gebruikt.
     
    De omgeving \verb|question| is aangepast zodat ze minimale header heeft (enkel een nummer), en dus in de praktijk binnen een oefening of voorbeeld kan gebruikt worden als alternatief voor \verb|itemize| of \verb|enumerate|.

    Binnen deze environments kunnen antwoorden worden gevraagd, en hints, feedback of oplossingen gegeven.
    \begin{itemize}
        \item \verb|answer|
        \item \verb|choice|
    \end{itemize}

    Zie verder voor meer uitleg, voorbeelden en afspraken voor het concrete gebruik van deze omgevingen in de zomercursus.

%    Todo:  uitleg voorbeelden  \verb|\answer| en \verb|\choice|

        
    \item \verb|hint|, \verb|feedback|, \verb|oplossing| 
    
    Kunnen worden gebruikt voor hints en feedback (\Smiley) (in het bijzonder \textit{binnen} bovenstaande omgevingen). 
    Er kan in beperkte mate worden aangegeven of, wanneer en hoe de hints en feedback wordt gegeven, zowel online als in pdf. 
    %Er is voor gekozen om in principe \verb|hint| te gebruiken en 
    Met \verb|\begin{oplossing}[toon]| wordt de oplossing altijd (dus ook in handout) getoond wordt. Dit is handig bij voorbeelden, waar de oplossing  online moet worden opengeklapt, maar die ook in de handouts volledig worden getoond.
    
    Detail: \verb|feedback| is enkel nuttig voor interactieve feedback, maar wordt intern ge(mis)bruikt om \verb|oplossing| te implementeren (als  (\verb|\begin{feedback}{solution}|).
    
       
    \item \verb|onlineOnly| : verschijnt enkel online (en, in de full pdf in het rood)
      
    Er is een tegenhanger, \verb|\pdfOnly|, die echter om onduidelijke redenen niet als environment is geïmplementeerd, maar als command; zie verder.
      
    \item \verb|prompt| :  verschijnt online en in niet-handout mode. Wordt niet gebruikt. 

    \item \verb|foldable| en \verb|expandable| : In HTML wordt dit in- of uitklapbaar; deze functionaliteit wordt door Ximera zelf gebruikt voor hints, en wordt allicht best NIET rechtstreeks gebruikt. De KULeuven versie van Ximera bevat \verb|accordion| en \verb|xmuitweiding| voor uitklapbare blokjes.

\end{enumerate}
    
\subsubsection{Commands}

\begin{enumerate}
    \item \verb|\pdfOnly|

     Tegenhanger van \textbackslash begin\{onlineOnly\}...\textbackslash end\{onlineOnly\}, maar om enigszins onduidelijke redenen als command in plaats van als environment. De inhoud verschijnt dus niet in de HTML. \verb|\pdfOnly| kan uitzonderlijk gebruikt worden voor constructies die in HTML niet (goed) werken.

    \item \verb|\youtube|, \verb|\geogebra|  en \verb|\desmos|
    
    Toont Youtube (voor \textit{opgegeven id}), Geogebra app of Desmos app (voor \textit{opgegeven link}).
    In de pdf wordt de link getoond (Todo: voeg ook QR-code toe)
    
    \item \verb|\graph| en \verb|\sageCell|
    
    Toont Desmos of Sage resultaat voor \textit{opgegeven code}.
    
\end{enumerate}

\subsubsection{Tikz}

In princiepe zijn alle tekeningen zijn gemaakt met Tikz. Dit betekent dat de broncode gewoon in de .tex bestanden staat, en dat dus ook de tekeningen eenvoudig aan te passen zijn. Een korte handleiding vind je bijvoorbeeld in  http://cremeronline.com/LaTeX/minimaltikz.pdf.

\subsection{Extra beperkingen van en door Ximera (t.o.v. zuiver \LaTeX-voor-pdf)}

Het gebruik van dezelfde \LaTeX-code om tegelijk zowel pdf als html te genereren, legt noodzakelijkerwijze een aantal beperkingen op. Zo is het in het algemeen niet mogelijk (of erg moeilijk) om in pdf en html bv. het horizontaal uitlijnen van tekst met behulp van blanco's (type \verb|\quad|) \textit{precies} dezelfde output te laten genereren. Laat staan manuele verticale \LaTeX fantasieën. Met de via Ximera automatisch beschikbare MathJAX ondersteuning, zijn wiskundige formules \textit{wel} quasi perfect automatisch transporteerbaar.

Het specifieke gebruik van Ximera legt bovendien nog enkele extra beperkingen op aan de \LaTeX functionaliteit. 

Merk op dat deze beperkingen in de meeste gevallen eerder \textit{voordelen} zijn dan \textit{nadelen}: u wordt als auteur/editor \textit{minder} verantwoordelijk voor de uiteindelijke layout dan voordien, en u kan zich dus \textit{meer} concentreren op de inhoud. Dat is dus een voordeel!

Significante aandachtspunten zijn:

\begin{enumerate}
	\item Nummering: om Ximera-technische redenen wordt de nummering in HTML en PDF onafhankelijk van elkaar berekend. Er is geprobeerd om ze zo veel mogelijk gelijk te maken, maar afwijkingen zijn mogelijk.
	\item Hyperlinks en  references: \verb|\ref| en \verb|\label| werken, maar \verb|\ref| WERKT NIET IN MATH-MODE; \verb|\link| geeft een voetnoot in PDF (wat enigszins onhandig is). \verb|\hyperref| is een handige manier om te refereren die zowel in PDF als online consistent werkt, onafhankelijk of de target-referentie al dan niet aanwezig is in de PDF.
	\item voetnoten: werken momenteel niet. De footnotemarkering (nummertje rechtsboven) wordt wel getoond, maar de link werkt niet. \verb|htlatex| genereert die footnote wel als aparte files, maar ze komen niet op de Ximera-webserver terecht. Met wat programmeerwerk (in \verb|xake|) kan dit eventueel worden opgelost.
	\item text-mode-in-math-mode werkt maar erg beperkt (bv. geen textit-in-text, dus 
    
    \verb|iets als $5+\text{ \textit{complicated} \tex-code } = 0$ | 
    
    werkt niet)  
	\item \verb|tabular| werkt onhandig: de tabel neemt per default online steeds de volledige breedte in. Via global.css is geprobeerd dat te verhelpen, maar the cure is worse than the problem (11/2020). Een veel gebruikt alternatief is \verb|array|, want dat werkt in mathmode, en dus in MathJAX, en dat gedraagt zich hier beter.

    \item er is een aparte \hyperref[xim:failcase]{lijst met constructies die niet werken}

\end{enumerate}

\subsection{Extra mogelijkheden en afspraken van, door en wegens Ximera}

Voor de \verb|activities| (dus .tex bestanden met \verb|documentclass{ximera}|) in de zomercursus worden volgende afspraken gemaakt

\paragraph{Preamble/layout}

Elke file begint met
\begin{verbatim}
\documentclass{ximera}
\input{../preamble.tex}
\addPrintStyle{..}

\begin{document}
    \author{Zomercursus Wiskunde}
    \xmtitle[Optionele pseudo-grappige abstract]{Dit is de Ximera Title}{Dit is de echte Ximera Abstract}
    \label{xim:vul_hier_de_bestandsnaam_in}
\end{verbatim}

De file preamble.tex bevat alle includes, newcommands etc. Wijzigingen in dit bestand hebben globale impact, op alle modules. Use with care.

Optioneel kunnen extra  adhoc-includes worden toegevoegd: \verb|\input{../preambles/functies.tex}|.

De macro \verb|\addPrintStyle{..}| zorgt ervoor dat de juiste printstyle wordt toegevoegd (enkel in PDF, met footers, etc)

De Ximera tex-files gebruiken voor layout best enkel de verder in deze tekst behandelde technieken. 
Dat garandeert een consistente layout in zowel PDF als online. 
Voor een cursus kan eventueel een aangepast printStyle.tex bestand worden voorzien, waarin voor de PDF versie op globale wijze de layout wordt bepaald (paginaformaat, lettertype, kleurgebruik, als dan niet kaders of gekleurde achtergrond voor definities en/of eigenschappen enzovoort.

Voor de online versie zijn dergelijke per-cursus aanpassingen mogelijk via CSS. Let wel op: precies dezelfde activiteiten (dus html-bestanden) kunnen door verschillende cursussen worden hergebruikt.

\subsubsection{Gebruikte environments}
Er wordt (buiten omgevingen voor oefeningen) enkel gebruik gemaakt van de volgende omgevingen:
   \begin{itemize}
   \item 'Theorie' (definities, eigenschappen etc., per default met groen-achtige layout)
   \begin{itemize} 
       \item \verb|definition|: voor definities
       \item \verb|notation|: om notaties uit te leggen
       \item \verb|axiom|: voor axioma's
       \item \verb|proposition|: voor eigenschappen
       \item \verb|theorem|: voor stellingen
       \item \verb|proof|: voor bewijzen
   \end{itemize}
   \item  Voorbeelden en oefeningen (per default met blauw-achtige layout)
   \begin{itemize}
       \item \verb|example|: voor voorbeelden
       \item \verb|exercise|: voor oefeningen

   \end{itemize}
   \item Opmerkingen en waarschuwingen(per default met geel-achtige layout)
   \begin{itemize} 
       \item \verb|remark|: voor opmerkingen
       \item \verb|warning|: voor waarschuwingen
   \end{itemize}
   \end{itemize}
Elke van deze environments is in preamble.tex gedefinieerd, en krijgt via \verb|printstyle.sty| (eventueel overschreven via \verb|printstyle_md.sty| of varianten) een eigen PDF-layout, en via \verb|global.css| (eventueel overschreven in \verb|zomercursusZ.css| of andere css files) een eigen HTML layout. 

Sommige expliciete layout voor pdf kan enkel in 'printstyles' gespecifieerd worden, want anders klaagt de compilatie naar HTML.

Volgende afspraken worden gemaakt voor het gebruik van de omgevingen:
\begin{itemize}
	\item er is in principe steeds een titel (gespecifieerd binnen vierkante haken)
	\item de eigenlijke inhoud start meestal direct na de header (truk: gebruik \verb|\nl| voor een newline enkel in PDF)
	\item de inhoud is self-contained (dus: de gebruikte symbolen worden \textit{binnen} de omgeving gedefinieerd) 
\end{itemize}

\begin{xmuitweiding}[Metavoorbeeld voor het gebruik van Voorbeelden]

Gebruik volgende code:

\begin{verbatim}
\begin{proposition}[Goed voorbeeld gebruik omgevingen]\nl
\label{vb:Gebruik_van_omgevingen_OK}  

Zij $x\in\{\text{alle Ximera omgevingen\}}.$
Dan geldt dat

\begin{align}
x & = \text{example} \implies x \text{ heeft als header Voorbeeld 1.2.}
\end{align} 
\end{proposition}

\end{verbatim}
voor volgende resultaat:
\begin{proposition}[Goed voorbeeld gebruik omgevingen]\nl \label{vb:Gebruik_van_omgevingen_OK} 
	
	Zij $x\in\{\text{alle Ximera omgevingen}\}.$
	Dan geldt dat
	
	\begin{align}
	 x & = \text{example} \implies x \text{ heeft als header Voorbeeld 1.2 }
	\end{align} 
\end{proposition}
en niet 

\begin{verbatim}
Zij $x\in\{\text{alle Ximera omgevingen\}}.$
Dan geldt dat

\begin{proposition}[Slecht voorbeeld gebruik omgevingen] 
\label{vb:Gebruik_van_omgevingen_NOK} \ 
	
	
	\begin{align}
	x & = \text{example} \implies x \text{ heeft als header Voorbeeld 1.2.}
	\end{align} 
\end{proposition}

\end{verbatim}

voor volgend resultaat:


Zij $x\in\{\text{alle Ximera omgevingen\}}.$
Dan geldt dat

\begin{proposition}[Slecht voorbeeld gebruik omgevingen] \label{vb:Gebruik_van_omgevingen_NOK} \ 
	\begin{align}
	x & = \text{example} \implies x \text{ heeft als header Voorbeeld 1.2.}
	\end{align} 
\end{proposition}

\end{xmuitweiding} 

\subsubsection{Oefeningen}

Ximera bevat 4 oefenomgevingen (\verb|exercise|, \verb|problem|, \verb|exploration| en \verb|question|) die standaard elk ongeveer dezelfde functionaliteit hebben. 

In de KULeuven versie van Ximera is er echter een specifiek gebruik voor de omgeving \verb|question|: een question heeft \textit{geen} header maar enkel een nummer. 
We gebruiken dus typisch een reeks van \verb|question|'s binnen een \verb|exercise| in plaats van een \verb|enumerate| of \verb|itemize|.
Omdat elke \verb|question| zijn \textit{eigen} feedback en hints kan hebben, kunnen de individuele vragen dus self-contained worden gemaakt. Een exercise of example \textit{kan} dan bestaan uit een lijst van questions (maar, er kan ook andere inhoud zijn in een exercise/example)

De omgevingen \verb|problem| en \verb|exploration| worden momenteel erg beperkt gebruikt. Het zou een idee kunnen zijn om \verb|exercise| te gebruiken voor 'inoefeningen' en \verb|problem| voor wat moeilijkere oefeningen. Dat is momenteel niet het geval.

Voorbeeld (vergelijk het verschil handout vs niet handout vs online):

{\footnotesize
\begin{verbatim}
\begin{example}[Afgeleiden van veeltermen] Bereken volgende afgeleiden:

	\begin{question}$\ddx(1+x+x^2) = \answer{1+2x}$ 
		\begin{oplossing}[toon] 
                $\ddx(1+x+x^2) \overset{\text{(somregel)}}{=} 
                \ddx 1+ \ddx x+ \ddx x^2 \overset{\mathrm{(rekenregels)}}{=} 
                0 + 1 + 2x = 1+ 2x$
        \end{oplossing}
	\end{question}

	\begin{question} $\ddx(1+x+x^2+x^3) = \answer[onlineshowanswerbutton]{1+2x+3x^2}$ 
		\begin{oplossing} 
           $\ddx(1+x+x^2 +x^3) = \ddx 1+ \ddx x+ \ddx x^2 \ddx x^3 
                               = 0 + 1 + 2x + 3x^2 = 1+ 2x + 3x^2$
			
			Maar je kan het resultaat van vorige oefening gebruiken: 
                $\ddx(1+x+x^2 +x^3) = \ddx(1+x+x^2) +\ddx x^3 =  1+ 2x + 3x^2$
			
			En je kan dit antwoord ook in één keer opschrijven, door term per term af te leiden:
			$\ddx(1+x+x^2 +x^3) = (0 +) 1+2x+3x^2$
		\end{oplossing}	
	\end{question}
    
	\begin{question}$\ddx(1+x+x^2+x^3+x^4) = \answer[onlinenoinput]{1+2x+3x^2+4x^3}$ 
    \end{question}
    
\end{example}

\end{verbatim}
}

geeft 

\begin{example}[Afgeleiden van veeltermen] Bereken volgende afgeleiden:

	\begin{question}$\ddx(1+x+x^2) = \answer{1+2x}$ 
		\begin{oplossing}[toon] 
                $\ddx(1+x+x^2) \overset{\text{(somregel)}}{=} 
                \ddx 1+ \ddx x+ \ddx x^2 \overset{\mathrm{(rekenregels)}}{=} 
                0 + 1 + 2x = 1+ 2x$
        \end{oplossing}
	\end{question}

	\begin{question} $\ddx(1+x+x^2+x^3) = \answer[onlineshowanswerbutton]{1+2x+3x^2}$ 
		\begin{oplossing} 
           $\ddx(1+x+x^2 +x^3) = \ddx 1+ \ddx x+ \ddx x^2 \ddx x^3 
                               = 0 + 1 + 2x + 3x^2 = 1+ 2x + 3x^2$
			
			Maar je kan het resultaat van vorige oefening gebruiken: 
                $\ddx(1+x+x^2 +x^3) = \ddx(1+x+x^2) +\ddx x^3 =  1+ 2x + 3x^2$
			
			En je kan dit antwoord ook in één keer opschrijven, door term per term af te leiden:
			$\ddx(1+x+x^2 +x^3) = (0 +) 1+2x+3x^2$
		\end{oplossing}	
	\end{question}
    
	\begin{question}$\ddx(1+x+x^2+x^3+x^4) = \answer[onlinenoinput]{1+2x+3x^2+4x^3}$ 
    \end{question}
    
\end{example}

% Zie bijvoorbeeld ook rekenen/machten.tex.   (TBV!)


\subsubsection{nFeAQ:  (not-)Frequently(-enough)-Asked-Questions}

\begin{accordion}
\begin{accordion-item}[Hoe maak ik een nieuwe xourse ?] \ 
	\begin{itemize}
		\item kies een bestandsnaam voor de nieuwe cursus (bv ECTS-code) (deze bestandsnaam komt enkel in de url en in de TeX-repo)
		\item kies een publieke naam voor de cursus (komt in de titel)
		\item kopieer een bestaande xourse, en pas die aan (bv door ximera's toe te voegen of te verwijderen)
		\item Let op volgende structuur:
		\begin{itemize}
			\item inhoud die min of meer per definitie \textit{enkel} voor de nieuwe xourse is, stop je in een folder met de naam van de xourse (bv, inleiding, layout, voorpagina, ...)
			\item gebruik x\_bestandsnaam voor dergelijke folder
		\end{itemize}
		\item voorbeelden: zomercursusZ.tex en G0N02B.tex
	\end{itemize}

\end{accordion-item}
\begin{accordion-item}[Hoe maak ik een nieuwe tabel ?] \ 
	\begin{itemize}
		\item je kopieert een bestaande tabel, en past die aan.
		\item \verb|tabular| heeft de onhebbelijkheid dat hij zich online over de volledig breedte uitsmeert. Voor 'smalle' tabellen is dat erg lelijk
		\item \verb|array| werk beter (maar is math-mode...; je kan beperkt text invoegen via \verb|\text|)
		\item je kan \verb|\tikz| gebruiken, en je tabel in een grafiek zetten
	\end{itemize}
	
\end{accordion-item}
\end{accordion}	

\end{document}
