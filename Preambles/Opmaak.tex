
\newcommand{\cocalc}{\includegraphics[scale=0.05]{../pictures/cocalc_icon.png}}              %afbeelding cocalc (verwijst naar voorbeelden/oefening in, cocalc

\newcommand{\pcode}[1]{\texttt{\textcolor{red}{#1}}} %python-commando's weergeven in deze stijl

\newcommand{\csv}[2]{\href{#1}{(\includegraphics[scale=0.05]{csv-icon.png} #2)}}

\newcommand{\fystip}{\begin{warning}
	Denk aan!
	\begin{itemize}
		\item correct notatie grootheden.
		\item bij een grootheid hoort een eenheid.
		\item resultaten in wetenschappelijke notatie.
		\item beduidende cijfers.
	\end{itemize}
\end{warning}}



% Instelling om stukjes code weer te geven
    \usepackage{listings}
    
    
    
    %New colors defined below
    \definecolor{codegreen}{rgb}{0,0.6,0}
    \definecolor{codegray}{rgb}{0.5,0.5,0.5}
    \definecolor{codepurple}{rgb}{0.58,0,0.82}
    \definecolor{backcolour}{rgb}{0.95,0.95,0.92}
    
    %Code listing style named "mystyle"
    \lstdefinestyle{mystyle}{
      backgroundcolor=\color{backcolour}, commentstyle=\color{codegreen},
      keywordstyle=\color{red},
      numberstyle=\tiny\color{codegray},
      stringstyle=\color{codepurple},
      basicstyle=\ttfamily\footnotesize,
      breakatwhitespace=false,         
      breaklines=true,                 
      captionpos=b,                    
      keepspaces=true,                 
      numbers=left,                    
      numbersep=5pt,                  
      showspaces=false,                
      showstringspaces=false,
      showtabs=false,                  
      tabsize=2
    }
    
    \lstset{style=mystyle}

%niveau oefeningen

%Specifiek element uit de lijst van kleuren




\newcommand{\oefniv}[1]
{
\ifnum #1=1 
    \def\mycolor{gray}
\else 
    \ifnum #1=2
        \def\mycolor{blue}
    \else
        \ifnum #1=3
            \def\mycolor{green}
        \else
            \ifnum #1=4 
                \def\mycolor{orange}
            \else
                \ifnum #1=5 
                    \def\mycolor{red}
                \else
                    \ifnum #1=6 
                        \def\mycolor{violet}
                    \else
                        \def\mycolor{black}
                    \fi
                \fi 
            \fi 
        \fi
    \fi 
\fi
\begin{tikzpicture}[scale=0.15]
    \draw[draw=white,fill=gray!50,thick] (0, 0) rectangle (2, 2);
    \draw[draw=white,fill=blue!50,thick] (2, 0) rectangle (4, 2);
    \draw[draw=white,fill=green!50,thick] (4, 0) rectangle (6, 2);
    \draw[draw=white,fill=orange!50,thick] (6, 0) rectangle (8, 2);
    \draw[draw=white,fill=red!50,thick] (8, 0) rectangle (10, 2);
    \draw[draw=white,fill=violet!50,thick] (10, 0) rectangle (12, 2);

    \draw[draw=white,fill=\mycolor,thick] ({(#1-1)*2}, 0) -- ({(#1-1)*2},4) -- ({(#1)*2},4) -- ({(#1)*2},0) -- cycle;
\end{tikzpicture}
\\
}

%eigenschappen veer

\usetikzlibrary{decorations.pathmorphing,calc,patterns}

\makeatletter

\def\pgfdecorationspringstraightlinelength{.5cm}
\def\pgfdecorationspringnumberofelement{8}
\def\pgfdecorationspringnaturallength{5cm}
\pgfkeys{%
  /pgf/decoration/.cd,
  spring straight line length/.code={%
    \pgfmathsetlengthmacro\pgfdecorationspringstraightlinelength{#1}},
  spring natural length/.code={%
    \pgfmathsetlengthmacro\pgfdecorationspringnaturallength{#1}},
  spring number of element/.store in=\pgfdecorationspringnumberofelement
}

\pgfdeclaredecoration{coil spring}{straight line}{%
  \state{straight line}[%
    persistent precomputation = {%
      % Compute the effective length of the spring (without the length
      % of the two straight lines): \pgfdecorationspringeffectivelength
      \pgfmathsetlengthmacro{\pgfdecorationspringeffectivelength}%
        {\pgfdecoratedpathlength-2*\pgfdecorationspringstraightlinelength}
      % Compute the effective length of one coil pattern:
      % \pgfdecorationspringeffectivelengthofonecoil
      \pgfmathsetlengthmacro{\pgfdecorationspringeffectivelengthofonecoil}%
        {\pgfdecorationspringeffectivelength/\pgfdecorationspringnumberofelement}
    },
    width = \pgfdecorationspringstraightlinelength,
    next state = draw spring]{%
      \pgfpathlineto{%
        \pgfqpoint{%
          \pgfdecorationspringstraightlinelength}{0pt}}
  }
  \state{draw spring}%
    [width=\pgfdecorationspringeffectivelengthofonecoil,
     repeat state=\pgfdecorationspringnumberofelement-1,next state=final]{%
       \pgfpathcurveto
         {\pgfpoint@onspringcoil{0    }{ 0.555}{1}}
         {\pgfpoint@onspringcoil{0.445}{ 1    }{2}}
         {\pgfpoint@onspringcoil{1    }{ 1    }{3}}
       \pgfpathcurveto
         {\pgfpoint@onspringcoil{1.555}{ 1    }{4}}
         {\pgfpoint@onspringcoil{2    }{ 0.555}{5}}
         {\pgfpoint@onspringcoil{2    }{ 0    }{6}}
       \pgfpathcurveto
         {\pgfpoint@onspringcoil{2    }{-0.555}{7}}
         {\pgfpoint@onspringcoil{1.555}{-1    }{8}}
         {\pgfpoint@onspringcoil{1    }{-1    }{9}}
       \pgfpathcurveto
         {\pgfpoint@onspringcoil{0.445}{-1    }{10}}
         {\pgfpoint@onspringcoil{0    }{-0.555}{11}}
         {\pgfpoint@onspringcoil{0    }{ 0    }{12}}
  }
  \state{final}{%
    \pgfpathlineto{\pgfpointdecoratedpathlast}
  }
}

\def\pgfpoint@onspringcoil#1#2#3{%
  \pgf@x=#1\pgfdecorationsegmentamplitude%
  \pgf@x=.5\pgf@x%
  \pgf@y=#2\pgfdecorationsegmentamplitude%
  \pgfmathparse{0.083333333333*\pgfdecorationspringeffectivelengthofonecoil}%
  \pgf@xa=\pgfmathresult pt
  \advance\pgf@x by#3\pgf@xa%
}


\makeatother

\tikzset{%
  Spring/.style = {%
    decoration = {%
      coil spring,
      spring straight line length = .5cm,
      % To be added
      spring natural length = #1,
      spring number of element = 8,
      amplitude=3mm},
    decorate,
    very thick},
  Spring/.default = {4cm}}
