% Extra pakketen



\usepackage{fontawesome5}
\usepackage{xcolor}
\usepackage{physics}
%\usepackage{siunitx}
\usepackage{lipsum}
%\usepackage{circuitikz}
\usepackage{csvsimple}
\usetikzlibrary{3d} % for canvas is
\usetikzlibrary{decorations.pathmorphing,decorations.markings} % for random steps & snake
\usepackage{datatool}
\usepackage{pgffor}
\usepackage{eurosym}
\usepackage{algorithmic}
\usepackage{pythontex} 
\usepackage{xparse}
%Extra commando's

\newcommand{\datum}{\today}
\newcommand{\nivcol}{black}
\newcommand{\iniv}{0} % index van het beheersingsniveau














% In plaats van grijs achtergrond, een rode achtergrond

%\renewcommand{\important}[1]{\ensuremath{\fcolorbox{red}{red!50}{$#1$}}}


% Uitbreidingwet zijn extra's die wel redelijkerwijze tot de leerstof van bv meer geavanceerde versies kunnen behoren voor de wetenschappelijke richtingen

\ifdefined\isXourse
   \ifdefined\xmuitweidingwet
   \else
       \def\xmnouitweidingwet{false}
   \fi
\fi

\ifdefined\xmuitweidingwet
\newcounter{xmuitweidingwet}  % anders error undefined ...  
\includecomment{xmuitweidingwet}
\else
\newtheoremstyle{dotless}{}{}{}{}{}{}{ }{}
\theoremstyle{dotless}
\newtheorem*{xmuitweidingwetnofrills}{}   % nofrills = no accordion; gebruikt dus de dotless theoremstyle!

\newcounter{xmuitweidingwet}
\newenvironment{xmuitweidingwet}[1][ ]%
{% 
	\refstepcounter{xmuitweidingwet}
	\begin{accordion}\begin{accordion-item}[\color{magenta}{{ $\ggg$ UITBREIDING WETENSCHAPPEN}} \arabic{xmuitweidingwet}: #1]%
			\begin{xmuitweidingwetnofrills}%
			}
			{\end{xmuitweidingwetnofrills}\end{accordion-item}\end{accordion}}   
\fi


%uitbreiding wiskunde


\ifdefined\isXourse
   \ifdefined\xmuitweidingwis
   \else
       \def\xmnouitweidingwis{false}
   \fi
\fi

\ifdefined\xmuitweidingwis
\newcounter{xmuitweidingwis}  % anders error undefined ...  
\includecomment{xmuitweidingwis}
\else
\newtheoremstyle{dotless}{}{}{}{}{}{}{ }{}
\theoremstyle{dotless}
\newtheorem*{xmuitweidingwisnofrills}{}   % nofrills = no accordion; gebruikt dus de dotless theoremstyle!

\newcounter{xmuitweidingwis}
\newenvironment{xmuitweidingwis}[1][ ]%
{% 
	\refstepcounter{xmuitweidingwis}
	\begin{accordion}\begin{accordion-item}[\color{purple}{{$\infty$ UITBREIDING WISKUNDE}} \arabic{xmuitweidingwis}: #1]%
			\begin{xmuitweidingwisnofrills}%
			}
			{\end{xmuitweidingwisnofrills}\end{accordion-item}\end{accordion}}   
\fi


% Oefening enkel voor wetenschappen en wiskunde


\newenvironment{oefwet}[1]
{   \textcolor{teal}{
    $\ggg WETENSCHAPPEN$ \hrulefill \\
        	#1 
        	}
}

% % afbeelding vak
\newcommand{\wis}[1]{$\infty$ #1}
\newcommand{\wet}[1]{ #1}
\newcommand{\fys}[1]{$\nabla$ #1}
\newcommand{\info}[1]{$\vdots$ #1}

\newcommand{\arch}[1]{$\measuredangle
$ #1}

\newcommand{\eco}[1]{\officialeuro #1}

%tabel grootheid en eenheid
% \newenvironment{eenheid}[4]
% {
% \renewcommand{\arraystretch}{2}

%     \begin{table}[h]
%         \centering
%         \begin{tabular}{|c|c||c|c|}
%         \hline
%              \large{Grootheid} &  \large{Symbool} & \large{Eenheid} & 
%              \large{Symbool} \\
%              \hline \hline
%              \cellcolor{teal!50}\textbf{\LARGE{#1}} & \cellcolor{teal!50}\textbf{\LARGE{#2}} & \cellcolor{teal!50}\textbf{\LARGE{#3}} & \cellcolor{teal!50}\textbf{\LARGE{#4}} \\
%              \hline
%         \end{tabular}
%     \end{table}

% \renewcommand{\arraystretch}{1}
% }